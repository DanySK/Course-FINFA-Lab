%%%%%%%%%%%%%%%%%%%%%%%%%%%%%%%%%%%%%%%%%%%%%%%%%%%%%%%%%%%%%%%%%%%%%%%%%%%%%%%%
\documentclass{beamer}
%
\usepackage[italian]{babel}
\usepackage[utf8]{inputenc}
\usepackage[T1]{fontenc}

%%%% Subtitle
%\subtitle[Short Subtitle]{Full Subtitle}
%%%% Authors
%%%% Institution/Partner
%%%% Date
% Structure slides
%
%%%%%%%%%%%%%%%%%%%%%%%%%%%%%%%%%%%%%%%%%%%%%%%%%%%%%%%%%%%%%%%%%%%%%%%%%%%%%%%%
\begin{document}
\title[Lab1 - FV]{Fondamenti di Informatica A \\ Laboratorio 4 \\ Programmazione strutturata}
\author[Danilo Pianini]{Danilo Pianini\\\texttt{danilo.pianini@unibo.it} \\ \vspace{3pt} Mirko Viroli\\\texttt{mirko.viroli@unibo.it} }
\institute[UNIBO]{\textsc{Alma Mater Studiorum}---Università di Bologna}
\date[\today]{\today}

\frame{\titlepage} 

\begin{frame}
\frametitle{Operazioni preliminari}
\begin{itemize}
 \item Accendere il computer
 \item Effettuare il log-in con le proprie credenziali istituzionali
  \begin{itemize}
    \item Normalmente sono simili a: \texttt{nome.cognome@studio.unibo.it}
  \end{itemize}
 \item Aprire Google Chrome, o altro browser
 \item Collegarsi al sito \url{http://bit.ly/finfa2013-e4}
 \item Scaricare l'esercitazione odierna
\end{itemize}
\end{frame}

\begin{frame}
\frametitle{Verifica del funzionamento del Framework F-S}
\begin{itemize}
 \item Sul desktop, troverete la cartella ``framework''
 \item Fate doppio click sulla cartella
 \item Si aprirà una nuova finestra, all'interno della quale troverete tre software
 \item Fare doppio click sull'icona del FrameworkFS
 \item Inserire nel campo ``Blocco da eseguire'' il testo \texttt{println(\textquotedbl{}Hello World\textquotedbl{});}
 \item Premere il tasto ``Esecuzione''
 \item Verificare che il risultato sia ``\texttt{Hello World}''
 \item In caso di risultato diverso dalle attese, chiamare subito il docente o il tutor
\end{itemize}
\end{frame}

\begin{frame}[fragile]
\frametitle{Come comportarsi di fronte agli errori}
\begin{itemize}
 \item Errare humanum est
  \begin{itemize}
    \item Ed è anche molto meglio sbagliare in laboratorio che all'esame!
  \end{itemize}
 \item È importante, di fronte agli errori, non scoraggiarsi, leggere \textbf{attentamente} il messaggio di errore, \textbf{capirlo} e studiare una contromisura.
 \item Java segnala gli errori in maniera piuttosto precisa...
 \item ...il framework, che ne usa una sottoparte molto ristretta, molto meno!
 \item In caso di errore, il framework vi mostrerà un messaggio che spiega cosa (il compilatore pensa che) sia successo, seguito dal codice Java reale che è stato invocato.
\end{itemize}
\end{frame}

\begin{frame}[fragile]
\frametitle{Istruzioni di questa esercitazione}
In questo laboratorio useremo quello che abbiamo imparato nel precedente. Sfrutteremo le nozioni di programmazione strutturata per costruire funzioni che risolvono problemi più complicati. Per ogni esercizio proposto, salvo diversa consegna, è \textbf{obbligatorio} eseguire, nell'ordine:
\begin{enumerate}
 \item Scrittura di un test che dimostri che avete compreso il problema da risolvere
 \item Scrittura di una strategia risolutiva in linguaggio naturale
 \item Se non si è certi di aver in mano una strategia corretta, interpellare il docente per avere un riscontro
 \item Tradurre la strategia in Java all'interno del framework F-S
 \item Interpellare il docente per commenti sul codice scritto
\end{enumerate}
\end{frame}

\begin{frame}[fragile]
\frametitle{Esempio per la grammatica \{1\}\{2|3\}}
\begin{block}{Descrizione in linguaggio naturale}
\begin{enumerate}
 \item Fintanto che trovo una sequenza di uno, vado avanti nell'array
 \item Fintanto che trovo dei due o dei tre, vado avanti nell'array
 \item Verifico di essere arrivato in fondo
\end{enumerate}
\end{block}
\begin{block}{Possibile test}
\scriptsize
 \begin{verbatim}
boolean test(){
   return f(new int[]{}) && 
   f(new int[]{1,2,3}) && 
   f(new int[]{1,1,1,1,1,1}) && 
   f(new int[]{2,3,2,2,2,2,3}) && 
   f(new int[]{1,2}) && 
   !f(new int[]{1,2,3,1}) && 
   !f(new int[]{2,3,5}) && 
   !f(new int[]{1,1,1,5,2,3,3}) && 
   !f(new int[]{0,1});
}
 \end{verbatim}
 \normalsize
\end{block}
\end{frame}

\begin{frame}[fragile]
\frametitle{Esempio per la grammatica \{1\}\{2|3\}}
\begin{block}{Traduzione in Java}
 \begin{verbatim}
boolean f(int[] a){
   int i = 0;
   /* Fintanto che trovo una sequenza di uno,
      vado avanti nell'array */
   for(; i < a.length && a[i] == 1; i++);
   /* Fintanto che trovo dei due o dei tre,
      vado avanti nell'array */
   for(; i < a.length && (a[i] == 2 || a[i] == 3); i++);
   /* Verifico di essere arrivato in fondo */
   return i == a.length;
}
 \end{verbatim}
\end{block}
Si risolve in tre righe!
\end{frame}

\begin{frame}[fragile]
\frametitle{Riconoscitori di grammatiche}
\begin{itemize}
 \item 5\{1|2|3|4\}5
 \item \{1 2 | 1 3\}
  \begin{itemize}
  \item \texttt{\{1,2,1,2,1,3\}} $\Rightarrow$ \texttt{true}
  \item \texttt{\{1,3\}} $\Rightarrow$ \texttt{true}
  \item \texttt{\{\}} $\Rightarrow$ \texttt{true}
  \item \texttt{\{0,1,2\}} $\Rightarrow$ \texttt{false}
  \item \texttt{\{12,13,12\}} $\Rightarrow$ \texttt{false}
  \item \texttt{\{12\}} $\Rightarrow$ \texttt{false}
  \end{itemize}
 \item $1^n$ 3 $2^n$
\end{itemize}
\end{frame}

\begin{frame}[fragile]
\frametitle{Riconoscitore di ordinamento}
 si realizzi la funzione \texttt{boolean grows(int[])}, che torna true se l'array è ordinato in maniera crescente. Alcuni esempi:
\begin{itemize}
 \item \texttt{\{0,1,2,3,4\}} $\Rightarrow$ \texttt{true}
 \item \texttt{\{4,4,4,4,4\}} $\Rightarrow$ \texttt{true}
 \item \texttt{\{\}} $\Rightarrow$ \texttt{true}
 \item \texttt{\{1,0,2,3,4\}} $\Rightarrow$ \texttt{false}
 \item \texttt{\{-1,0,2,3,4\}} $\Rightarrow$ \texttt{true}
 \item \texttt{\{-1,0,2,3,2\}} $\Rightarrow$ \texttt{false}
\end{itemize}
\end{frame}

\begin{frame}[fragile]
\frametitle{Prodotto scalare}
 si realizzi la funzione \texttt{double dotProduct(double[], double[])}, che dati due array di double di pari lunghezza calcola analiticamente il loro prodotto scalare. Alcuni esempi:
\begin{itemize}
 \item \texttt{\{0.5,0.6,0.9\}, \{0.5,0.6,0.9\}} $\Rightarrow$ \texttt{1.42}
 \item \texttt{\{4,4,4,4,4\}, \{1,2,3,4,5\}} $\Rightarrow$ \texttt{60}
 \item \texttt{\{\}, \{\}} $\Rightarrow$ \texttt{0}
 \item \texttt{\{10\}, \{10\}} $\Rightarrow$ \texttt{100}
 \item \texttt{\{10\}, \{10, 20\}} $\Rightarrow$ \texttt{0}
\end{itemize}
Ossia, il risultato $r$, dati due array $a$ e $b$ lunghi $length$, è:
$$
r = \sum_{i=0}^{length-1}{a[i]b[i]}
$$
\end{frame}

\begin{frame}[fragile]
\frametitle{Normalizzatore}
 Si realizzi la funzione \texttt{double normalise(int[])}, che dato un array, calcola il valor medio diviso la differenza fra il valore massimo ed il valore minimo. Il risultato deve essere preciso. Alcuni esempi:
\begin{itemize}
 \item \texttt{\{0,1,2,3,4\}} $\Rightarrow$ \texttt{0.5}
 \item \texttt{\{4,4,4,4,4\}} $\Rightarrow$ \texttt{1}
 \item \texttt{\{\}} $\Rightarrow$ \texttt{0}
 \item \texttt{\{1,10,21,3,4\}} $\Rightarrow$ \texttt{0.39}
\end{itemize}
Ossia, risultato $r$, dato un array $a$ lungo $a.length$, è:
$$
r = \frac{\frac{\sum_{i=0}^{a.length-1}a[i]}{a.length}}{max(a)-min(a)}
$$
Se $max(a) > min(a)$, 1 altrimenti.

Suggerimento: nonostante la soluzione ottima sia realizzabile scorrendo l'array una sola volta (un solo ciclo \texttt{for}), l'esercizio si risolve più facilmente se si realizzano separatamente le funzioni che calcolano il minimo ed il massimo di un array.
\end{frame}

\begin{frame}[fragile]
\frametitle{Esercizi aggiuntivi}
Chi termina in anticipo i sei esercizi proposti, può accedere agli esercizi opzionali che seguono.

La difficoltà alcuni di questi è superiore a quelli d'esame. In ogni caso, se riuscite a risolverli agevolmente, siete ad un ottimo livello!
\end{frame}

\begin{frame}[fragile]
\frametitle{Esercizi aggiuntivi: grammatiche}
\begin{itemize}
 \item \{0[1]\}0
 \item \{2\}3
 \item \{2\}3\{2\}
 \item 5\{1\{2|3\}4\}5
 \item $1^n$ $2^n$
 \item $1^n$ 3 $2^{2n}$
 \item $1^n$ 3 $2^n$ 4 $5^n$
 \item $1^n$ 3 \{$2^n$\} 4 [$5^n$]
 \item \{$1^n$ $2^n$\}
\end{itemize}
\end{frame}

\begin{frame}[fragile]
\frametitle{Funzioni su array e numeri}
\begin{itemize}
 \item \texttt{double avgFromTo(int[], int, int)}: calcola la media esatta dei valori di un array all'interno di un intervallo.
 \item \texttt{int max(int[])}: calcola il massimo valore dell'array.
 \item \texttt{boolean ballot(boolean[])}: torna true se l'array contiene più valori true che false, false altrimenti.
 \item \texttt{boolean isPrime(int)}: torna true se il numero in ingresso è primo.
 \item \texttt{int nextPrime(int)}: torna il numero primo successivo a quello dato. È consentito l'uso di \texttt{boolean isPrime(int)} realizzata in precedenza.
\end{itemize}
\end{frame}

\begin{frame}
\frametitle{Al termine dell'esercitazione}
\begin{itemize}
 \item Ricordarsi di effettuare il logout
  \begin{itemize}
    \item Dal menu Start, selezionare ``Disconnetti''
  \end{itemize}
 \item Una volta disconnessi:
  \begin{itemize}
    \item Se siete del primo turno, lasciate le macchine accese, di modo che i vostri colleghi del turno successivo siano agevolati
    \item Se siete del secondo turno, spegnete le macchine
  \end{itemize}
\end{itemize}
\end{frame}


\end{document}

