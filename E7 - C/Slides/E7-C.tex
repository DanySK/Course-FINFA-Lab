%%%%%%%%%%%%%%%%%%%%%%%%%%%%%%%%%%%%%%%%%%%%%%%%%%%%%%%%%%%%%%%%%%%%%%%%%%%%%%%%
\documentclass{beamer}
%
\usepackage[italian]{babel}
\usepackage[utf8]{inputenc}
\usepackage[T1]{fontenc}

%%%% Subtitle
%\subtitle[Short Subtitle]{Full Subtitle}
%%%% Authors
%%%% Institution/Partner
%%%% Date
% Structure slides
%
%%%%%%%%%%%%%%%%%%%%%%%%%%%%%%%%%%%%%%%%%%%%%%%%%%%%%%%%%%%%%%%%%%%%%%%%%%%%%%%%
\begin{document}
\title[Lab1 - FV]{Fondamenti di Informatica A \\ Laboratorio 7 \\ Rudimenti di C}
\author[Danilo Pianini]{Danilo Pianini\\\texttt{danilo.pianini@unibo.it} \\ \vspace{3pt} Pietro Brunetti \\\texttt{p.brunetti@unibo.it} \\ \vspace{3pt} Mirko Viroli\\\texttt{mirko.viroli@unibo.it} }
\institute[UNIBO]{\textsc{Alma Mater Studiorum}---Università di Bologna}
\date[\today]{\today}

\frame{\titlepage} 

\begin{frame}
\frametitle{Operazioni preliminari}
\begin{itemize}
 \item Accendere il computer
 \item Effettuare il log-in con le proprie credenziali istituzionali
  \begin{itemize}
    \item Normalmente sono simili a: \texttt{nome.cognome@studio.unibo.it}
  \end{itemize}
 \item Aprire Google Chrome, o altro browser
 \item Collegarsi al sito \url{http://campus.unibo.it/185639/}
 \item Scaricare istruzioni (slides) e codice
 \item Scompattare il pacchetto in una cartella a piacimento: questa sarà la vostra directory di lavoro di oggi.
 \item All'interno del pacchetto, troverete degli esempi di codice del prof. Viroli. Prendetevi del tempo per osservarli, cercate di capire cosa fanno, ed eventualmente utilizzateli come ispirazione per svolgere gli esercizi.
\end{itemize}
\end{frame}

\begin{frame}
\frametitle{Elementi base del file system}
\begin{itemize}
 \item I sistemi operativi odierni consentono di memorizzare permanentemente le informazioni su supporti di memorizzazione di massa (dischi magnetici, dispositivi a stato solido), unità ottiche (CD, DVD, Blu-Ray), memory stick, ecc...
 \item Le informazioni su questi supporti sono organizzate in files e cartelle:
  \begin{itemize}
    \item i files contengono le informazioni;
    \item le cartelle sono contenitori, all'interno contengono i files ed altre cartelle.
  \end{itemize}
 \item La cartella più esterna, che contiene tutte le altre, è detta root. Essa è il livello gerarchico più alto nel file system.
  \begin{itemize}
    \item In Windows, ciascun file system ha come root una lettera di unità (e.g. \texttt{C:}, \texttt{D:}).
    \item In *nix (Linux, MacOS, BSD, Solaris, etc), vi è una unica radice, ossia \texttt{/}
  \end{itemize}
 \item La stringa che descrive un intero percorso dalla root fino ad un elemento del file system prende il nome di directory (e.g. \texttt{C:\textbackslash{}Windows\textbackslash{}win32.dll}, \texttt{/home/user/frameworkFS.jar})
\end{itemize}
\end{frame}

\begin{frame}[fragile]
\frametitle{Manipolare il file system}
L'utente può osservare e manipolare il file system:
\begin{itemize}
 \item sapere quali files e cartelle contiene una cartella
 \item creare nuovi files e cartelle
 \item spostare file e cartelle dentro altre cartelle
 \item rinominare files e cartelle
 \item eliminare files e cartelle
\end{itemize}
Il software deputato ad osservare e manipolare il file system prende il nome di \alert{file manager}.
\begin{itemize}
 \item Su Windows, esso è ``Esplora risorse'' (\texttt{explorer.exe})
 \item Su MacOS, il principale è ``Finder''
 \item Su Linux (e Android) ne esistono diversi (Nautilus, Dolphin, Thunar, Astro...)
\end{itemize}
Provate ad effettuare le seguenti operazioni (utilizzando il file manager):
\begin{enumerate}
 \item Creare due cartelle, di nome \texttt{f1} ed \texttt{f2}
 \item Rinominare \texttt{f2} in \texttt{f12}
 \item Spostare \texttt{f12} dentro \texttt{f1}
\end{enumerate}
\end{frame}

\begin{frame}[fragile]
\frametitle{Il terminale}
Il terminale consente di impartire comandi al computer tramite testo.
\begin{itemize}
 \item Nell'antichità (in termini informatici) le interfacce grafiche erano sostanzialmente inesistenti, e l'interazione con i calcolatori avveniva di norma tramite interfaccia testuale
 \item Tutt'oggi, le interfacce testuali sono utilizzate:
\begin{itemize}
 \item per automatizzare le operazioni
 \item per velocizzare le operazioni (scrivere un comando è spesso molto più veloce di andare a fare click col mouse in giro per lo schermo)
 \item per fare operazioni complesse con pochi semplici comandi
 \item non tutti i software sono dotati di interfaccia grafica
 \item alcune opzioni di configurazione del sistema operativo restano accessibili solo via linea di comando
\begin{itemize}
 \item (anche su Windows: ad esempio i comandi per associare le estensioni ad un eseguibile)
\end{itemize}
\end{itemize}
\end{itemize}
\end{frame}

\begin{frame}[fragile]
\frametitle{Struttura di un comando}
\begin{block}{Tutti i comandi condividono una struttura comune}
\texttt{commandName parameters arguments} 
\end{block}
\begin{itemize}
 \item Solitamente i parametri vengono prefissi da un simbolo \texttt{-} o \texttt{-{}-}, o (in ambiente Windows) \texttt{/}
\end{itemize}
\begin{block}{Alcuni esempi (da capire, non da eseguire!)}
\texttt{java -jar frameworkFS.jar} 
\scriptsize
\begin{itemize}
 \item \textbf{Cosa fa?} Avvia una macchina virtuale java, chiedendo l'esecuzione del file ``frameworkFS.jar''
 \item \textbf{Nome comando:} \texttt{java}
 \item \textbf{Parametri:} \texttt{-jar}
 \item \textbf{Argomenti:} \texttt{frameworkFS.jar}
\end{itemize}
\normalsize
\texttt{rm -r -f /home/user/uselessData /home/user/other} 
\scriptsize
\begin{itemize}
 \item \textbf{Cosa fa?} Rimuove le directory \texttt{/home/user/uselessData} e \texttt{/home/user/other}, tutte le cartelle in esse contenute e tutti i file in esse contenute.
 \item \textbf{Nome comando:} \texttt{rm} (sta per remove)
 \item \textbf{Parametri:} \texttt{-f} (sta per ``force''), \texttt{-r} (sta per ``recursive'')
 \item \textbf{Argomenti:} \texttt{/home/user/uselessData}, \texttt{/home/user/other}
\end{itemize}
\end{block}
\normalsize
\end{frame}

\begin{frame}[fragile]
\frametitle{Aprire un terminale in laboratorio}
\begin{itemize}
 \item In laboratorio, troverete il terminale (prompt dei comandi) clickando su Start $\Rightarrow$ Programmi $\Rightarrow$ Accessori $\Rightarrow$ Prompt dei comandi
 \item Metodo più rapido: Start  $\Rightarrow$ Nella barra di ricerca, digitare \texttt{cmd} $\Rightarrow$ clickare su \texttt{cmd.exe}
\end{itemize}
\end{frame}

\begin{frame}[fragile]
\label{slide:commands}
\frametitle{Manipolare il file system dal terminale}
\begin{center}
  \begin{tabular}{| l | c | c |}
    \hline
    \textbf{Operazione} & \textbf{Comando Unix} & \textbf{Comando Win} \\ \hline
    \scriptsize{}Visualizzare la directoy corrente & \texttt{pwd} & \texttt{echo \%cd\%}  \\ \hline
    \scriptsize{}Eliminare il file \texttt{f} (non va con le cartelle!) & \texttt{rm f} & \texttt{del f} \\ \hline
    \scriptsize{}Eliminare la directory \texttt{nd} & \texttt{rm -r nd} & \texttt{rd nd} \\ \hline
    \scriptsize{}Contenuto della directory corrente & \texttt{ls -l} & \texttt{dir} \\ \hline
    \scriptsize{}Avviare un eseguibile \texttt{p} & \texttt{./p} & \texttt{.\textbackslash{}p} \\ \hline
    \scriptsize{}Cambiare unità disco (passare a D:) & -- & \texttt{D:} \\ \hline
    \scriptsize{}Passare alla directory \texttt{nd} & \texttt{cd nd} & \texttt{cd nd} \\ \hline
    \scriptsize{}Passare alla directory di livello superiore & \texttt{cd ..} & \texttt{cd..} \\ \hline
    \scriptsize{}Spostare un file \texttt{f1} in \texttt{f2} & \texttt{mv f1 f2} & \texttt{move f1 f2} \\ \hline
    \scriptsize{}Copiare il file \texttt{f} in \texttt{fc} & \texttt{cp f fc} & \texttt{copy f fc} \\ \hline
    \scriptsize{}Creare la directory \texttt{d} & \texttt{mkdir d} & \texttt{md d} \\ \hline
  \end{tabular}
\end{center}
Eseguire delle prove ed esser certi di aver compreso come utilizzare ogni comando. Per \emph{cominciare} l'esame, in particolare, dovrete usare il comando \texttt{cd}: siate certi di aver capito cosa fa!
\end{frame}

\begin{frame}[fragile]
\frametitle{Uso intelligente del terminale}
\begin{block}{Autocompletamento}
\scriptsize{}
Sia Unix che Windows offrono la possibilità di effettuare autocompletamento, ossia chiedere al sistema di provare a completare un comando. Per farlo si utilizza il tasto ``tab'' (quello con due frecce orientate in maniera opposta, sopra il lucchetto). Si testi la funzione di autocompletamento sul proprio sistema operativo.
\end{block}
\begin{block}{Memoria dei comandi precendenti}
\scriptsize{}
Sia Unix che Windows offrono la possibilità di richiamare rapidamente i comandi inviati precedentemente premendo il tasto ``freccia su''. I sistemi Unix supportano anche il lancio di comandi eseguiti in sessioni precedenti (se inviate un comando e riavviate il sistema, e al riavvio premete ``freccia su'', vi apparirà il comando precedente. 
\end{block}
\begin{block}{Interruzione di un programma}
\scriptsize{}
È possibile interrompere forzatamente un programma (ad esempio perché inloopato). Per farlo, sia su Windows che in Unix, si prema ctrl+c.
\end{block}
\begin{block}{Ricerca nella storia dei comandi precedenti}
\scriptsize{}
Premendo ctrl+r seguito da un testo da cercare, i sistemi Unix supportano la ricerca all'interno dei comandi lanciati recentemente, anche in sessioni utente precedenti. Non disponibile su Windows.
\end{block}
\end{frame}

\begin{frame}
\frametitle{compilazione di un file C}
\begin{itemize}
 \item Utilizzando il comando \texttt{gcc} visto a lezione, si compili il file \texttt{compile\_me.c}
 \item \alert{Si consiglia di usare sempre l'opzione \texttt{-Wall}, per ricevere tutti i warning generati dal vostro programma.}
 \item Utilizzando l'opzione \texttt{-o} del compilatore, si produca l'eseguibile \texttt{UselessProgram}
 \item Si controlli il sorgente del programma prevedendo il suo funzionamento, e poi lo si esegua nel seguente modo:
 \begin{itemize}
  \item Sotto Windows, scrivendo una delle seguenti:
  \begin{itemize}
    \item UselessProgram.exe
    \item UselessProgram
    \item .\textbackslash{}UselessProgram
    \item .\textbackslash{}UselessProgram.exe
  \end{itemize}
  \item Sotto UNIX (MacOS, Linux), scrivendo:
  \begin{itemize}
    \item ./UselessProgram
  \end{itemize}
 \end{itemize}
\end{itemize}
\end{frame}

\begin{frame}
\frametitle{La tua nuova migliore amica: \texttt{void printf()}}
\begin{itemize}
 \item Si crei un nuovo file \texttt{helloworld.c}, il cui \texttt{main()} utilizzi la funzione \texttt{printf} per stampare a video la stringa ``Hello, world!'', quindi vada a capo.
  \begin{itemize}
    \item Si compili con il nome di \texttt{HelloWorld} e si esegua.
  \end{itemize}
 \item Si crei un nuovo file \texttt{evens.c}, il cui \texttt{main()} utilizzi la funzione \texttt{printf} per stampare a video i primi 500 numeri pari separati da uno spazio, quindi vada a capo.
  \begin{itemize}
    \item Si compili con il nome di \texttt{Evens} e si esegua.
  \end{itemize}
\end{itemize}
Nota: per utilizzare la funzione \texttt{printf}, è necessario includere i prototipi delle funzioni fornite dalla libreria di sistema \texttt{stdio}. Per farlo, utilizzare la direttiva \texttt{\#include <stdio.h>}
\end{frame}

\begin{frame}
\frametitle{La tua prima funzione in C}
Si costruisca in C la funzione con prototipo \texttt{int mcm(int, int)}, che dati in ingresso due interi torna il loro minimo comune multiplo. Si costruisca una funzione \texttt{int test(void)} che ne verifichi il funzionamento, e si usi il main per stampare a video il risultato di tale test. Si compili come \texttt{mcm} e si esegua.
 \begin{itemize}
  \item Si noti che, dati due numeri a e b, il loro mcm si ottiene, ad esempio, sommando al più piccolo dei due, ad ogni iterazione, il suo valore iniziale, fin quando i due numeri sono uguali.
  \item Si costruisca mcm in modo che depositi subito a e b in due nuove variabili a0 e b0. A questo punto si esegua un for che ad ogni passo aggiunge al più piccolo fra a e b il corrispondente a0 o b0. Tale for terminerà quando a e b sono uguali fra loro, e a quel punto si ritornerà o a o b.
  \item Esempio di mcm(2,3):
  \begin{enumerate}
  \scriptsize
   \item a = 2, b = 3, a0 = 2, b0 = 3;
   \item a < b $\rightarrow{}$ a = 4, b = 3, a0 = 2, b0 = 3;
   \item a > b $\rightarrow{}$ a = 4, b = 6, a0 = 2, b0 = 3;
   \item a < b $\rightarrow{}$ a = 6, b = 6, a0 = 2, b0 = 3;
   \item a == b $\rightarrow{}$ si ritorna 6;
  \end{enumerate}
 \end{itemize}
\end{frame}

\begin{frame}
\frametitle{La tua seconda funzione in C}
Con riferimento alla funzione che realizza il fattoriale vista a lezione:
\begin{itemize}
 \item Si modifichi la funzione in maniera tale che utilizzi il tipo di dato \texttt{unsigned long int}
 \item Si definisca tramite macro \texttt{MAX\_VAL}, che rappresenta il più grande valore che, passato in ingresso alla funzione, porta a computare un valore corretto
  \begin{itemize}
    \item Per trovare il valore, si potrebbe ad esempio iniziare a stampare diversi fattoriali, vedendo per quale valore per primo si ha overflow.
  \end{itemize}
 \item Si modifichi la funzione in maniera tale che, qualora il valore passato come parametro sia maggiore di \texttt{MAX\_VAL}, essa ritorni 0
 \item Si scriva una appropriata funzione \texttt{int test(void)} che ne verifica il funzionamento
 \item Si faccia in modo che il main esegua il test e stampi a video ``\texttt{OK}'' se il test passa, \texttt{``ERROR''} altrimenti.
\end{itemize}
\end{frame}

\begin{frame}
\frametitle{Scrittura di funzioni matematiche: \texttt{math.h}}
La libreria math.h contiene una serie di funzioni matematiche pronte all'uso, si veda \url{http://www.cplusplus.com/reference/cmath/}. La si usi per costruire all'interno di un file \texttt{functions.c} le seguenti funzioni:
\begin{itemize}
 \item \texttt{double ex(double x)} --- calcola $e^x$
 \item \texttt{double sinx(double x)} --- calcola $sin(x)$
 \item \texttt{double tanx(double x)} --- calcola $tan(x)$
 \item \texttt{double logx(double x)} --- calcola $log(x)$
\end{itemize}
Queste funzioni non devono far altro che chiamare quelle di \texttt{math.h} al loro interno.
\end{frame}

\begin{frame}
\frametitle{Creazione di una libreria di funzioni}
Si implementino inoltre le seguenti:
\begin{itemize}
 \item \texttt{int minThree(int a, int b, int c)} --- torna il più piccolo di tre interi
 \item \texttt{int randInInterval(int a, int b)} --- crea un numero casuale compreso fra \texttt{a} e \texttt{b}. Si utilizzi la funzione \texttt{int rand(void)}, fornita in stdlib.h (si veda \url{http://www.cplusplus.com/reference/cstdlib/})
 \item \texttt{int previousSquare(int x)} --- calcola il più grande quadrato perfetto minore o uguale ad \texttt{x}
 \item \texttt{double solveEquation(double a, double b, double c)} --- funzione che calcola una soluzione dell'equazione $ax^2 + bx + c$, tornando NaN se non vi sono soluzioni
 \begin{itemize}
  \item Si noti che le soluzioni di tale equazione sono  ovviamente $x = \frac{-b \pm \sqrt{b^2-4ac}}{2a}$
  \item Per tornare NaN, si può utilizzare la macro NAN definita in \texttt{math.h}
 \end{itemize}
\end{itemize}
\end{frame}

\begin{frame}
\frametitle{Scrittura di un header file}
Si crei un header file \texttt{functions.h} contenente tutti i prototipi delle funzioni realizzate in precedenza. Come riferimento, si consideri il file \texttt{filters.h} allegato al codice.

Si includa l'header appena creato in \texttt{functions.c}, utilizzando la direttiva: \\
\texttt{\#include \textquotedbl{}functions.h\textquotedbl{}}

Si compili il file in maniera tale da ottenere una libreria (opzione \texttt{-c}). Si ricordi di utilizzare la flag \texttt{-lm}, che indica al compilatore di caricare la libreria \texttt{math.h}.
\end{frame}

\begin{frame}
\frametitle{Uso della libreria creata}
\begin{itemize}
 \item Si crei un nuovo file \texttt{uselib.c}
 \item Si includa il file header appena creato 
 \item Si scriva in uselib.c una funzione \texttt{int main()} che utilizzi al suo interno tutte le funzioni della libreria precedente, stampando a video i risultati delle prove che esegue.
 \item Si compili, ricordandosi di includere \texttt{functions.o} fra i file di libreria, e si esegua il programma.
 \item Dato che questo programma utilizza la libreria \texttt{functions.h}, che a sua volta utilizza \texttt{math.h}, è necessario utilizzare la flag \texttt{-lm} quando si compila!
\end{itemize}
\end{frame}

\begin{frame}[fragile]
\frametitle{Uso di \texttt{void printf()}}
Si realizzi quanto segue:
\scriptsize
\begin{itemize}
 \item Un programma, \texttt{zeromatrix.c}, che stampi a video una matrice 30x30 di zeri. Per farlo, si realizzi una funzione \texttt{void printEmptyMatrix(int n)} che stampa a video una matrice nxn di zeri.
 \item Un programma, \texttt{matr.c}, che stampi a video una matrice 30x30 in cui tutti i valori sono 4. Per farlo, si realizzi una funzione \texttt{void printMatrix(int n, int v)} che stampa a video una matrice nxn di \texttt{v}. 
 \item Un programma, \texttt{morematr.c}, che stampi a video una matrice 30x30 in cui le colonne pari hanno valore 4 e quelle dispari valore 6. Per farlo, si realizzi una funzione \texttt{void printMatrixCols(int n, int v1, int v2)} che stampa a video una matrice nxn le cui colonne pari hanno valore v1 e quelle dispari valore v2.
 \item Un programma, \texttt{arrow.c}, che stampi una freccia fatta di 0, come mostrato nell'esempio seguente. Per farlo, si realizzi una funzione \texttt{void printArrow(int n)} che stampa a video una freccia con coda lunga \texttt{n}. L'esempio sotto è per \texttt{n} = 8.
\end{itemize}
\begin{verbatim}
                  0
          0000000000
          00000000000
          0000000000
                  0
\end{verbatim}
\end{frame}
\end{document}

