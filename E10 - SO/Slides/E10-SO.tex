%%%%%%%%%%%%%%%%%%%%%%%%%%%%%%%%%%%%%%%%%%%%%%%%%%%%%%%%%%%%%%%%%%%%%%%%%%%%%%%%
\documentclass{beamer}
%
\usepackage[italian]{babel}
\usepackage[utf8]{inputenc}
\usepackage[T1]{fontenc}

%%%% Subtitle
%\subtitle[Short Subtitle]{Full Subtitle}
%%%% Authors
%%%% Institution/Partner
%%%% Date
% Structure slides
%
%%%%%%%%%%%%%%%%%%%%%%%%%%%%%%%%%%%%%%%%%%%%%%%%%%%%%%%%%%%%%%%%%%%%%%%%%%%%%%%%
\begin{document}
\title[Lab1 - FV]{Fondamenti di Informatica A \\ Laboratorio 10 \\ Accesso al file system}
\author[Danilo Pianini]{Danilo Pianini\\\texttt{danilo.pianini@unibo.it} \\ \vspace{3pt} Mirko Viroli\\\texttt{mirko.viroli@unibo.it} }
\institute[UNIBO]{\textsc{Alma Mater Studiorum}---Università di Bologna}
\date[\today]{\today}

\frame{\titlepage} 

\begin{frame}[fragile]
\frametitle{Operazioni preliminari}
\begin{itemize}
 \item \textbf{AVVISO:} per chi volesse sperimentare con i device file ed i comandi in stile Unix presentati a lezione, può fare boot sulla \textbf{macchina fisica}, dove è disponibile una macchina virtuale Linux installata in \texttt{D:\textbackslash{}finfa}
 \item La stessa macchina sarà utilizzata per l'ultimo laboratorio
 \item Slides: \url{http://bit.ly/finfa2013-e10}
 \item Codice: \url{http://bit.ly/finfa2013-e10-code}
\end{itemize}
\end{frame}

\begin{frame}[fragile]{Linguaggio Assembly}
Conoscere il linguaggio assembly non fa parte delle competenze che dovete acquisire per questo corso, ma è utile per capire come funziona la macchina sottostante (approfondirete in Calcolatori Elettronici!).
\begin{itemize}
 \item Si consideri \texttt{sum.c}. Se ne comprenda il funzionamento, quindi si compili con: \\ \texttt{gcc -S sum.c -o sum -Wall} \\ Quali files sono stati prodotti?
 \item Si apra e si analizzi \texttt{sum.s}. Si cerchino delle corrispondenze fra il codice C e quello ASM:
  \begin{itemize}
    \item Come sono organizzate le funzioni?
    \item Come sono organizzati i loop?
%    \item Cosa accade prima dell'esecuzione di \texttt{call}, istruzione ASM che trasferisce il controllo ad un'altra funzione?
  \end{itemize}
\end{itemize}
\end{frame}

\begin{frame}[fragile]{Esempio di accesso al file system: \texttt{cat}}
\begin{itemize}
 \item Si osservi il contenuto di \texttt{cat.c}. Cosa realizza?
 \item Una volta compreso il funzionamento, si compili e si esegua, ad esempio con \\ \texttt{./cat cat.c filter.c}
 \item Si modifichi cat.c perché sia in grado di funzionare con più di un file in input, ossia: \\ \texttt{./cat file1 file2 file3} \\ restituisce in output \\ \texttt{<contenuto di file1> \\ <contenuto di file2> \\ <contenuto di file3>}
 \item \textbf{SUGGERIMENTI:}
 \begin{enumerate}
 \scriptsize
  \item si crei una nuova funzione \\ \texttt{void printFile(char *)} \\ che, dato una stringa rappresentante il path di un file, lo stampa a schermo. Si noti che questa funzionalità è già offerta da \texttt{cat.c} all'interno del \texttt{main}.
 \scriptsize
  \item A questo punto, si chiami la funzione per ogni argomento passato, costruendo un ciclo \texttt{for} che scorre \texttt{argv} dal secondo elemento in poi e richiami la funzione costruita sopra.
 \scriptsize
  \item Si elimini il messaggio di warning riguardante i parametri in eccesso.
 \end{enumerate}
\end{itemize}
\end{frame}

\begin{frame}[fragile]{Filtro input-output, pipe, redirect}
\begin{itemize}
 \item Si osservi il contenuto di \emph{filter.c}. Cosa realizza?
 \item Si compili e si testi il programma
 \item Si immagini che Twitter abbia messo a pagamento l'uso delle vocali: si modifichi \texttt{filter.c} affinché sostituisca a tutte le vocali della stringa in input la lettera Y, in modo da poter scrivere dei tweet il cui costo di pubblicazione sia zero.
 \item Si provi ad utilizzare il comando con la pipe: \\ \texttt{echo look at the pipe! \textbar{} filter} \\ Come si comporta?
 \item Si provi ora ad utilizzare l'operatore di redirect per creare un nuovo file contenente il tweet: \\ \texttt{echo look at the pipe! \textbar{} filter > tweet} \\ Cosa succede?
\end{itemize}
\end{frame}

\begin{frame}[fragile]{Salvataggio di dati}
Si scriva un nuovo programma che:
\begin{itemize}
 \item Legga due argomenti e li converta in numeri, interpretandoli come valore inizio e valore fine.
 \item Generi, ad intervalli di 0.01 a partire dal valore inizio e fino al valore fine, i valori della funzione coseno.
 \item Conservi questi valori all'interno di un apposito array
 \item Salvi tali valori in un file \texttt{data.log}. Il formato dei dati deve essere analogo a quello del file \texttt{data.log} fornito nei code examples.
\end{itemize}
\scriptsize
\textbf{SUGGERIMENTI} --- Si ricorda che:
\begin{itemize}
 \item la funzione \texttt{double atof(const char*)} converte una stringa in double, si trova in \texttt{stdlib.h}
 \item la funzione \texttt{int fprintf(FILE *, const char *, ...)} in \texttt{stdio.h} consente di stampare su un descrittore di file del testo formattato (come \texttt{printf}, ma su qualunque file).
 \item esiste la funzione \texttt{double cos(double x)} di \texttt{math.h}.
 \item per compilare un sorgente che includa \texttt{math.h}, è necessario specificare la flag \texttt{-lm} a \texttt{gcc} (e.g. \texttt{gcc -lm -Wall programma.c -o programma})
 \item Si osservi la slide 27 del pacchetto riguardante la gestione dei file!
\end{itemize}
\end{frame}

\begin{frame}[fragile]{Caricamento di dati}
Si scriva un nuovo programma che:
\begin{itemize}
 \item Legga un file \texttt{data.log} formattato come il file \texttt{data.log} allegato ai code examples.
 \item Inserisca i dati caricati all'interno di un apposito array (\texttt{double *})
 \item Stampi a video il contenuto dell'array
%  \item \textbf{BONUS:} calcolare correttamente la dimensione dell'array
\end{itemize}
\scriptsize
\textbf{SUGGERIMENTI} --- Si ricorda che:
\begin{itemize}
 \item la funzione \texttt{int fscanf(FILE *, const char *, ...)} legge da uno stream qualsiasi una stringa formattata ed inserisce i risultati nelle aree di memoria puntate dagli argomenti aggiuntivi passati
 \item la funzione \texttt{void rewind(FILE *)} riporta il file descriptor passato all'inizio del file
\end{itemize}
\end{frame}

\begin{frame}[fragile]{Opzionali}
Svolgete gli esercizi seguenti se avete completato tutta la precedente parte di esercitazione.
\end{frame}

\begin{frame}[fragile]{Modifica di \texttt{cp}: \texttt{mv}}
\begin{itemize}
 \item Si osservi il contenuto di \texttt{cp.c}. Cosa realizza?
 \item Sulla base di \texttt{cp.c} e \texttt{cat.c}, si realizzi un nuovo programma \texttt{mv.c} tale che:
 \begin{itemize}
  \item Il programma invece di copiare il file lo sposti, ossia tale che la copia originale venga cancellata. Si utilizzi la funzione \\ \texttt{int remove(const char *)} di \texttt{stdlib.h}, che dato un nome di file lo elimina. Si veda \url{http://www.cplusplus.com/reference/cstdio/remove/}
  \item Il programma non utilizzi un array predimensionato come in cp.c, ma calcoli la dimensione del file (come in mv.c) e crei un array della dimensione appropriata. Si ricorda che in C per allocare una quantità di memoria nota solo a runtime è necessario l'uso della funzione \\ \texttt{void* malloc (unsigned int size);}
 \end{itemize}
\end{itemize}
\end{frame}

\begin{frame}[fragile]{\texttt{replace}}
\begin{itemize}
 \item Si osservi il contenuto del filtro \texttt{replace.c}. Cosa realizza?
 \item Sulla base di \texttt{replace.c}, si realizzi un nuovo programma \texttt{duplicate.c} tale che:
 \begin{itemize}
  \item Il programma è un filtro che prende in ingresso un solo argomento, del quale considera il primo carattere
  \item ogni volta che in ingresso il programma incontra il simbolo passato, deve duplicarlo.
  \item \textbf{ESEMPIO:} \\ \texttt{echo casa | ./duplicate a} \\ deve avere come output: \\ \texttt{caasaa}
 \end{itemize}
\end{itemize}
\end{frame}

\end{document}

