%%%%%%%%%%%%%%%%%%%%%%%%%%%%%%%%%%%%%%%%%%%%%%%%%%%%%%%%%%%%%%%%%%%%%%%%%%%%%%%%
\documentclass{beamer}
%
\usepackage[italian]{babel}
\usepackage[utf8]{inputenc}
\usepackage[T1]{fontenc}

%%%% Subtitle
%\subtitle[Short Subtitle]{Full Subtitle}
%%%% Authors
%%%% Institution/Partner
%%%% Date
% Structure slides
%
%%%%%%%%%%%%%%%%%%%%%%%%%%%%%%%%%%%%%%%%%%%%%%%%%%%%%%%%%%%%%%%%%%%%%%%%%%%%%%%%
\begin{document}
\title[Lab1 - FV]{Fondamenti di Informatica A \\ Laboratorio 8 \\ Array e puntatori in C}
\author[Danilo Pianini]{Danilo Pianini\\\texttt{danilo.pianini@unibo.it} \\ \vspace{3pt} Mirko Viroli\\\texttt{mirko.viroli@unibo.it} }
\institute[UNIBO]{\textsc{Alma Mater Studiorum}---Università di Bologna}
\date[\today]{\today}

\frame{\titlepage} 

\begin{frame}
\frametitle{Operazioni preliminari}
\begin{itemize}
 \item Accendere il computer
 \item Effettuare il log-in con le proprie credenziali istituzionali
  \begin{itemize}
    \item Normalmente sono simili a: \texttt{nome.cognome@studio.unibo.it}
  \end{itemize}
 \item Aprire Google Chrome, o altro browser
 \item Collegarsi al sito \url{http://bit.ly/finfa2013-e8}
 \item Scaricare il pacchetto \url{http://bit.ly/finfa2013-e8-code}
 \item Scompattare il pacchetto in una cartella a piacimento: questa sarà la vostra directory di lavoro di oggi.
\end{itemize}
\end{frame}

\begin{frame}
\frametitle{Comprensione di un esempio con puntatori}
\begin{itemize}
 \item Aprire il file \texttt{e2.c} contenente la funzione \texttt{swap} e un \texttt{main} che effettua un semplice test di essa: se ne analizzi il codice cercando di capirne il funzionamento.
 \item Prima di compilare il codice, si provi a capire quale dovrebbe essere l'output prodotto sulla console.
 \item Compilare ed eseguire. Si verifichi che l'output del programma corrisponde a quanto previsto.
\end{itemize}
\end{frame}

\begin{frame}
\frametitle{Modifica dell'esempio}
Si modifichi \texttt{e2.c} come segue:
\begin{itemize}
 \item Si elimini il contenuto del \texttt{main}.
 \item Si crei un nuovo array di interi, contenente i valori 10, 20, 30 e 40.
 \item Si stampi il valore dei quattro elementi dell'array tramite apposita \texttt{printf}
 \item Si invochi la funzione \texttt{swap} passando come primo argomento il puntatore al primo valore dell'array e come secondo valore il puntatore al secondo elemento dell'array (suggerimento: l'accesso all'i-esimo elemento dell'array a si effettua tramite \texttt{a[i]}, mentre il puntatore all'i-esimo elemento di a si può ottenere con \texttt{\&a[i]})
 \item Si stampi nuovamente il valore dei quattro elementi dell'array tramite apposita \texttt{printf}
 \item Si compili ed esegua: il risultato è conforme alle attese?
\end{itemize}
\end{frame}

\begin{frame}
\frametitle{\texttt{void *malloc(unsigned int size)}}
La funzione \texttt{void *malloc(unsigned int size)} consente di allocare per il programma una porzione di memoria grande \texttt{size} bytes.
\begin{itemize}
 \item Si osservi il contenuto di \texttt{create\_array.c}. Cosa fa? Una volta compreso il funzionamento, si compili e si esegua.
 \item Si modifichi \texttt{create\_array.c} affinché crei un nuovo array il cui contenuto sia numerato progressivamente (e.g. \texttt{create\_array(5)} deve tornare \texttt{\{1,2,3,4,5\}}. Si modifichi il \texttt{main} perché esegua un test sulla funzione.
\end{itemize}
\end{frame}

\begin{frame}
\frametitle{Stringhe di caratteri in C}
Diversamente da Java, C non ha supporto per il tipo \texttt{String} a livello di linguaggio. In C, le Stringhe sono in realtà dei \texttt{char *}, terminati dal carattere speciale \texttt{\textbackslash{}0}.
\begin{itemize}
 \item Utilizzando anche la funzione malloc vista in precedenza, e facendo riferimento ai lucidi visti a lezione, si realizzi una funzione che concatena due stringhe, separandole con uno spazio (si noti che la concatenazione semplice è già risolta nei lucidi!)
 \item Si scriva un \texttt{main} che testi la funzione appena realizzata
 \item Si scriva una funzione \texttt{char *scanToLast(char *s)} che, data una stringa di lunghezza non nota, torna un puntatore all'ultimo elemento. Non è consentito l'uso di \texttt{strlen}. La si testi nel \texttt{main}.
\end{itemize}
\end{frame}

\begin{frame}
\frametitle{Output negli argomenti}
I puntatori consentono di utilizzare argomenti come output: si passa un puntatore all'area in cui vogliamo il risultato, la funzione computa e scrive in quell'area il risultato, in questo modo il chiamante può accedervi.
\begin{itemize}
 \item Si realizzi la funzione\\ \texttt{void sum(int a, int b, int *res)} \\che dati due interi ne calcola la somma e la mette nell'area di memoria puntata da res. Si verifichi nel \texttt{main} il funzionamento.
 \item Si realizzi la funzione\\ \texttt{void arrayCreator(int size, int **res)} \\che crea un array di interi lungo \texttt{size} nell'area di memoria puntata da res, utilizzando la \texttt{malloc}, quindi lo riempie con valori crescenti a partire da 0. Si verifichi nel \texttt{main} il funzionamento.
\end{itemize}
\end{frame}

\begin{frame}
\frametitle{Esercizi aggiuntivi}
Chi ha terminato gli esercizi di base, può dar libero sfogo alle proprie abilità svolgendo quelli aggiuntivi.
\end{frame}

\begin{frame}
\frametitle{Riconoscitori di grammatiche in C}
Si risolvano gli esercizi proposti nei file \texttt{e3.c} ed \texttt{e4.c}, sull'implementazione di un semplice riconoscitore di grammatiche e della sua versione ricorsiva.
\end{frame}

\begin{frame}
\frametitle{Matrici in C}
\begin{itemize}
 \item Si osservi il file \texttt{matr.c}. Se ne comprenda il funzionamento, e solo dopo averlo compreso si compili e si esegua.
 \item Sulla base di matr.c, si crei un nuovo programma in \texttt{matrBorder.c}, che crea una matrice della dimensione desiderata, quindi ne resetta il bordo esterno (prima e ultima riga, prima e ultima colonna).
\end{itemize}
\end{frame}

\begin{frame}
\frametitle{Output negli argomenti}
Perché fermarsi ai puntatori a puntatore quando si possono fare puntatori a puntatori a puntatori? (e via per induzione...)
\begin{itemize}
 \item Si realizzi all'interno di matr.c la funzione\\ \texttt{void matrixCreator(int size, int ***res)} \\che crea una matrice quadrata di interi grande \texttt{size} nell'area di memoria puntata da res, utilizzando la \texttt{malloc}, quindi lo riempie con valori crescenti a partire da 0. Si usi nel \texttt{main} questa funzione al posto di \\ \texttt{int **matr(int size,int elem)}.
\end{itemize}
\end{frame}


\end{document}

