%%%%%%%%%%%%%%%%%%%%%%%%%%%%%%%%%%%%%%%%%%%%%%%%%%%%%%%%%%%%%%%%%%%%%%%%%%%%%%%%
\documentclass{beamer}
%
\usepackage[italian]{babel}
\usepackage[utf8]{inputenc}
\usepackage[T1]{fontenc}

%%%% Subtitle
%\subtitle[Short Subtitle]{Full Subtitle}
%%%% Authors
%%%% Institution/Partner
%%%% Date
% Structure slides
%
%%%%%%%%%%%%%%%%%%%%%%%%%%%%%%%%%%%%%%%%%%%%%%%%%%%%%%%%%%%%%%%%%%%%%%%%%%%%%%%%
\begin{document}
\title[Lab1 - FV]{Fondamenti di Informatica A \\ Soluzioni Laboratorio 2 \\ Funzioni e ricorsione}
\author[Danilo Pianini]{Danilo Pianini\\\texttt{danilo.pianini@unibo.it} \\ \vspace{3pt} Mirko Viroli\\\texttt{mirko.viroli@unibo.it} }
\institute[UNIBO]{\textsc{Alma Mater Studiorum}---Università di Bologna}
\date[\today]{\today}

\frame{\titlepage} 

\begin{frame}
\frametitle{Ricorsione}
\begin{itemize}
 \item \texttt{int f(int n)\{ return n==0? 1 : n*f(n-1); \}}
\end{itemize}
\begin{itemize}
 \item Realizza il fattoriale.
 \item Fino a \texttt{f(12)}, incluso. Poi gli interi non sono abbastanza grandi da contenere il risultato (tant'è che \texttt{f(13)/f(12)} dà come risultato \texttt{4}, e non \texttt{13}.
 \item Utilizzando il tipo \texttt{long}, si riesce a migliorare: \\ \texttt{long f(long n)\{ return n==0? 1 : n*f(n-1); \}}. Si arriva a computare correttamente \texttt{f(20)}
 \item Non tail: l'ultima operazione da effettuare non è la chiamata ricorsiva ma la somma.
\end{itemize}
\end{frame}

\begin{frame}[fragile]
\frametitle{Ricorsione}
Si analizzi la funzione \texttt{f(int, int)}:
\begin{verbatim}
int f(int x, int y) {
   return f(x, y, x, y);
}

int f(int x, int y, int z, int w) {
   return z==w ? z :
      z>w ? f(x, y, z, w+y) :
         f(x, y, z+x, w);
}
\end{verbatim}
\begin{itemize}
 \item Realizza il minimo comune multiplo
 \item Ricorsione tail: in ogni caso, l'ultima operazione effettuata è la chiamata ricorsiva.
\end{itemize}
\end{frame}


\begin{frame}[fragile]
\frametitle{Modifica di funzioni}
\begin{verbatim}
int sum(int[] array){
   return array.length > 4 ? 0 : sum(array, 0);
}
int sum(int[] array,int pos){
   return (pos==array.length) ? 0 :
      array[pos]+sum(array,pos+1);
}

boolean test_sum(){
   return sum(new int[]{10,20,30})==60 &&
      sum(new int[]{10,20,30,40,50})==0 &&
      sum(new int[]{10})==10 &&
      sum(new int[]{})==0 &&
      sum(new int[]{-10,-20,-30})==-60;
}
\end{verbatim}

\end{frame}
\begin{frame}[fragile]
\frametitle{Modifica di funzioni}
Si consideri \texttt{sum(int[])}, nella versione non modificata.
\begin{itemize}
 \item Vengono sommati solo gli elementi in posizione pari, ma si rischia di andare fuori dall'array.
 \item Sostituendo \texttt{==} con \texttt{>=} si risolve il problema dell'uscita dall'array, poiché non appena l'indice esce dall'intervallo consentito si ritorna 0 (che viene sommato all'indietro con i valori precedenti).
\end{itemize}
\end{frame}

\begin{frame}[fragile]
\frametitle{Costruzione di funzioni a partire da funzioni esistenti}
\begin{verbatim}
int sum_odd(int[] array){
   return sum_odd(array, 1);
}

int sum_odd(int[] array, int pos){
   return (pos>=array.length) ? 0 :
      array[pos] + sum_odd(array, pos + 2);
}

boolean test_sum_odd(){
   return sum_odd(new int[]{10,20,30})==20 &&
      sum_odd(new int[]{10,20,30,40,50})==60 &&
      sum_odd(new int[]{10,20,30,40})==60 &&
      sum_odd(new int[]{10,20})==20 &&
      sum_odd(new int[]{10})==0 &&
      sum_odd(new int[]{})==0 &&
      sum_odd(new int[]{-10,-20,-30})==-20;
}
\end{verbatim}
\end{frame}

\begin{frame}[fragile]
\frametitle{Costruzione di funzioni a partire da funzioni esistenti}
\begin{verbatim}
int sum_abs(int[] array) {
   return sum_abs(array, 0);
}

int sum_abs(int[] array, int pos) {
   return (pos==array.length) ? 0 :
      (array[pos] < 0 ? -array[pos] : array[pos])
      + sum_abs(array, pos + 1);
}

boolean test_sum_abs() {
   return sum_abs(new int[]{10,20,30})==60 &&
      sum_abs(new int[]{10,20,-30,-40,50})==150 &&
      sum_abs(new int[]{-10,20,30,40})==100 &&
      sum_abs(new int[]{-10,-20})==30 &&
      sum_abs(new int[]{10})==10 &&
      sum_abs(new int[]{})==0;
}
\end{verbatim}
\end{frame}

\begin{frame}[fragile]
\frametitle{Esercizi aggiuntivi}
Da questo punto, vengono proposti esercizi aggiuntivi. Sono consigliati per coloro che hanno completato \textbf{tutti} i precedenti con successo e desiderano esercitarsi ulteriormente.
\end{frame}

\begin{frame}[fragile]
\frametitle{Ricorsione}
Si analizzi la funzione \texttt{f(int, int)}:
\begin{verbatim}
int f(int x) { return x == 0 ? 0 : f2(x-1); }
int f2(int x) { return x == 1 ? 1 : f(x-1); }
\end{verbatim}
\begin{itemize}
 \item Con \texttt{f(5)}, si ha uno stack overflow: le due funzioni si richiamano l'un l'altra all'infinito, riempiendo l'area di memoria riservata a contenere lo stack.
\end{itemize}
\end{frame}

\begin{frame}[fragile]
\frametitle{Progettare nuove funzioni}
\begin{verbatim}
int length(int[] array) {
   return array.length;
}

boolean test_length() {
   return length(new int[]{10,20,30})==3 &&
      length(new int[]{10,20,-30,-40,50})==5 &&
      length(new int[]{-10,20,30,40})==4 &&
      length(new int[]{-10,-20})==2 &&
      length(new int[]{10})==1 &&
      length(new int[]{})==0;
}
\end{verbatim}
\end{frame}

\begin{frame}[fragile]
\frametitle{Progettare nuove funzioni}
\begin{verbatim}
int evens(int[] array) {
   return evens(array, 0, 0);
}

int evens(int[] array, int pos, int buf) { 
   return pos == array.length ? buf :
      evens(array, pos + 1, buf + (array[pos] % 2 == 0 ? 1 : 0));
}

boolean test_evens() {
   return evens(new int[]{12,22,32})==3 &&
      evens(new int[]{1,2,-3,-4,5})==2 &&
      evens(new int[]{-1,2,3,4})==2 &&
      evens(new int[]{-1,-2})==1 &&
      evens(new int[]{10})==1 &&
      evens(new int[]{})==0;
}
\end{verbatim}
\end{frame}

\begin{frame}[fragile]
\frametitle{Progettare nuove funzioni}
\begin{verbatim}
boolean arrayEquals(int[] a, int[] b) {
return a.length != b.length ? false : arrayEquals(a, b, 0);
}

boolean arrayEquals(int[] a, int[] b, int pos) {
   return pos == a.length ? true :
      a[pos] != b[pos] ? false : 
      arrayEquals(a, b, pos + 1);
}

boolean test_arrayEquals() {
   return arrayEquals(new int[]{}, new int[]{}) &&
   arrayEquals(new int[]{1,2,3}, new int[]{1,2,3}) &&
   arrayEquals(new int[]{1,0,0,87}, new int[]{1,0,0,87})&&
   !arrayEquals(new int[]{}, new int[]{0}) &&
   !arrayEquals(new int[]{0,1,2}, new int[]{2,1,0}) &&
   !arrayEquals(new int[]{0,0,0,0}, new int[]{0,0,0,1});
}
\end{verbatim}
\end{frame}

\begin{frame}[fragile]
\frametitle{Progettare nuove funzioni: riconoscitori EBNF}
\begin{verbatim}
boolean ric21(int[] a) {
   return ric2(a, 0);
}

boolean ric2(int[] a, int pos) {
   return pos == a.length ? false :
      a[pos] != 2 ? ric1(a, pos) : 
      ric2(a, pos + 1);
}

boolean ric1(int[] a, int pos) {
   return a[pos] == 1 && a.length == pos + 1;
}
\end{verbatim}
\end{frame}

\begin{frame}[fragile]
\frametitle{Progettare nuove funzioni: riconoscitori EBNF}
\begin{verbatim}
/* Note that 1{1}1 = 11{1} */
boolean ric111(int[] a) {
   return a.length > 1 && a[0] == 1 && a[1] == 1 ?
      ric111(a, 1) : false;
}

boolean ric111(int[] a, int pos) {
   return pos == a.length ||
      (a[pos] == 1 && ric111(a, pos+1));
}
\end{verbatim}
\end{frame}

\begin{frame}[fragile]
\frametitle{Progettare nuove funzioni: riconoscitori EBNF}
\begin{verbatim}
boolean ric12(int[] a) {
   return a.length == 0 || ric12(a, 0);
}

boolean ric12(int[] a, int pos) {
   return pos == a.length ? true : 
      (a[pos] == 1 || a[pos] == 2) && ric12(a, pos + 1);
}
\end{verbatim}
\end{frame}

\begin{frame}[fragile]
\frametitle{Progettare nuove funzioni: riconoscitori EBNF}
\begin{verbatim}
boolean ric123(int[] a) {
   return a.length > 1 && a[0] == 1 && ric23(a, 1);
}
boolean ric23(int[] a, int pos) {
   return pos == a.length ? false : a[pos] == 2 ? ric23(a, pos + 1) :
      ric3(a, pos);
}
boolean ric3(int[] a, int pos) {
   return pos == a.length -1 && a[pos] == 3;
}
\end{verbatim}
\end{frame}

\begin{frame}[fragile]
\frametitle{Progettare nuove funzioni: riconoscitori EBNF}
\begin{verbatim}
boolean ric1234(int[] a) {
   return a.length == 0 || ric12(a, 0);
}
boolean ric12(int[] a, int pos) {
   return pos == a.length ||
      ((a[pos] == 1 || a[pos] == 2) ? ric12(a, pos + 1) :
      ric34(a, pos));
}
boolean ric34(int[] a, int pos) {
   return pos == a.length ||
      ((a[pos] == 3 || a[pos] == 4) ? ric34(a, pos + 1) :
      false);
}
\end{verbatim}
\end{frame}

% \begin{frame}[fragile]
% \frametitle{Avanzato: ottimizzazione tail}
% Soluzione: si calcola il risultato mano a mano che si procede, accumulandolo in una variabile che viene passata quando si effettua la chiamata.
% \begin{verbatim}
% int sum(int[] array){
%    return sum(array, 0, 0);
% }
% int sum(int[] array,int pos, int buf){
%    return pos==array.length ? buf :
%       sum(array, pos+1, buf + array[pos]);
% }
% 
% int sum_odd(int[] array){
%    return sum_odd(array, 1, 0);
% }
% int sum_odd(int[] array,int pos, int buf){
%    return pos>=array.length ? buf :
%       sum(array, pos+2, buf + array[pos]);
% }
% \end{verbatim}
% \end{frame}
% 
% \begin{frame}[fragile]
% \frametitle{Avanzato: ottimizzazione tail}
% Soluzione: si calcola il risultato mano a mano che si procede, accumulandolo in una variabile che viene passata quando si effettua la chiamata.
% \begin{verbatim}
% int sum_abs(int[] array){
%    return sum_abs(array, 0, 0);
% }
% int sum_abs(int[] array,int pos, int buf){
%    return pos==array.length ? buf : 
%       sum_abs(array, pos+1, buf +
%          (array[pos] >= 0 ? array[pos] :
%             -array[pos]));
% }
% \end{verbatim}
% \end{frame}


\end{document}

