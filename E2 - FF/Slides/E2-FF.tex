%%%%%%%%%%%%%%%%%%%%%%%%%%%%%%%%%%%%%%%%%%%%%%%%%%%%%%%%%%%%%%%%%%%%%%%%%%%%%%%%
\documentclass{beamer}
%
\usepackage[italian]{babel}
\usepackage[utf8]{inputenc}
\usepackage[T1]{fontenc}

%%%% Subtitle
%\subtitle[Short Subtitle]{Full Subtitle}
%%%% Authors
%%%% Institution/Partner
%%%% Date
% Structure slides
%
%%%%%%%%%%%%%%%%%%%%%%%%%%%%%%%%%%%%%%%%%%%%%%%%%%%%%%%%%%%%%%%%%%%%%%%%%%%%%%%%
\begin{document}
\title[Lab1 - FV]{Fondamenti di Informatica A \\ Laboratorio 2 \\ Funzioni e ricorsione}
\author[Danilo Pianini]{Danilo Pianini\\\texttt{danilo.pianini@unibo.it} \\ \vspace{3pt} Mirko Viroli\\\texttt{mirko.viroli@unibo.it} }
\institute[UNIBO]{\textsc{Alma Mater Studiorum}---Università di Bologna}
\date[\today]{\today}

\frame{\titlepage} 

\begin{frame}
\frametitle{Operazioni preliminari}
\begin{itemize}
 \item Accendere il computer
 \item Effettuare il log-in con le proprie credenziali istituzionali
  \begin{itemize}
    \item Normalmente sono simili a: \texttt{nome.cognome@studio.unibo.it}
  \end{itemize}
 \item Aprire Google Chrome, o altro browser
 \item Collegarsi al sito \url{http://bit.ly/finfa2013-e2}
 \item Scaricare l'esercitazione odierna
\end{itemize}
\end{frame}

\begin{frame}
\frametitle{Verifica del funzionamento del Framework F-F}
\begin{itemize}
 \item Sul desktop, troverete la cartella ``framework''
 \item Fate doppio click sulla cartella
 \item Si aprirà una nuova finestra, all'interno della quale troverete tre software
 \item Fare doppio click sull'icona del FrameworkFF
 \item Inserire nel campo ``Espressioni da valutare'' il testo \texttt{$1+2$}
 \item Premere il tasto ``Valutazione''
 \item Verificare che il risultato sia ``3''
 \item In caso di risultato diverso dalle attese, chiamare subito il docente o il tutor
\end{itemize}
\end{frame}

\begin{frame}
\frametitle{Come comportarsi di fronte agli errori}
\begin{itemize}
 \item Errare humanum est
  \begin{itemize}
    \item Ed è anche molto meglio sbagliare in laboratorio che all'esame!
  \end{itemize}
 \item È importante, di fronte agli errori, non scoraggiarsi, leggere \textbf{attentamente} il messaggio di errore, \textbf{capirlo} e studiare una contromisura.
 \item Java segnala gli errori in maniera piuttosto precisa...
 \item ...il framework, che ne usa una sottoparte molto ristretta, molto meno!
 \item In caso di errore, il framework vi mostrerà un messaggio che spiega cosa (il compilatore pensa che) sia successo, seguito dal codice Java reale che è stato invocato.
\end{itemize}
\end{frame}

\begin{frame}
\frametitle{Come esercitarsi con il Framework}
\begin{itemize}
 \item Questo framework è una versione potenziata di FV
 \item Oltre a consentire di valutare espressioni, consente di definire funzioni
 \item Le funzioni definite potranno poi essere utilizzate all'interno dell'espressione da valutare
\end{itemize}
\end{frame}

\begin{frame}
\frametitle{Struttura di una funzione}
Partiamo da un semplice esempio:\\ \texttt{int s(int i)\{ return i+1;\}}
\begin{itemize}
 \item \texttt{int s(int i)} è detta ``signature''
  \begin{itemize}
    \item \texttt{La signature dà tutte le informazioni necessarie all'identificazione di una funzione:}
  \begin{itemize}
    \item nome della funzione
    \item numero e tipo dei parametri di ingresso
    \item tipo del parametro di ritorno
  \end{itemize}
  \item Conoscendo la signature di una funzione, è possibile utilizzarla!
  \end{itemize}
 \item \texttt{return i+1} è detto ``corpo''
  \begin{itemize}
    \item Il corpo contiene l'effettiva implementazione
    \item Nel corpo si trovano le istruzioni che vengono effettivamente svolte quando si invoca la funzione
  \end{itemize}
\end{itemize}
\end{frame}

\begin{frame}
\frametitle{Invocazione delle funzioni}
\begin{itemize}
 \item Si inserisca la funzione \texttt{int s(int i)} della slide precedente nel box in alto
 \item Senza inserire nulla nel secondo box, si valuti. Perché questo risultato?
 \item Si provi ad intuire, e quindi si verifichi, il risultato della valutazione delle seguenti:
\begin{itemize}
 \item \texttt{s(0)}
 \item \texttt{s(10)}
 \item \texttt{s(2147483647)} --- Perché?
 \item \texttt{s(0)+s(0)}
 \item \texttt{s(s(0))}
\end{itemize}
\end{itemize}
\end{frame}

\begin{frame}
\frametitle{Correttezza semantica, singola funzione}
Una funzione è semanticamente corretta se:
\begin{itemize}
 \item I parametri sono usati in maniera compatibile con il loro tipo
 \item Il tipo dell'espressione corpo è uguale, oppure convertibile per coercizione, al tipo di ritorno della funzione
 \item Si provi ad intuire, e quindi si verifichi, quali delle seguenti funzioni sono semanticamente corrette. Ove presenti, si leggano e si capiscano i messaggi di errore.
\begin{itemize}
 \item \texttt{int f(int x, int y)\{ return x+y;\}}
 \item \texttt{int f(double x, int y)\{ return x+y;\}}
 \item \texttt{float f(double x, int y)\{ return x+y;\}}
 \item \texttt{int f(int x, int y)\{ return x+\textquotedbl{}y\textquotedbl{};\}}
 \item \texttt{double f(double x, long y)\{ return x+y;\}}
 \item \texttt{int f(boolean b, boolean b2)\{return (b\&{}b2)?0:1f;\}}
 \item \texttt{int f(boolean b, boolean b2)\{return (b\&{}b2)?-1:1;\}}
 \item \texttt{boolean f(int x, int y)\{ return x==y;\}}
\end{itemize}
\end{itemize}
\end{frame}

\begin{frame}
\frametitle{Correttezza dell'invocazione}
Una funzione è correttamente invocata se:
\begin{itemize}
 \item I parametri passati sono uguali (o convertibili per coercizione) al tipo dichiarato
 \item Il tipo di ritorno è uguale (o convertibile per coercizione) al tipo da associare.
 \item Si considerino le seguenti funzioni corrette: 
\begin{itemize}
 \item \texttt{boolean f3(int x,int y)\{ return x==y;\}}
 \item \texttt{boolean f4(double x,double y)\{ return x!=y;\}}
\end{itemize}
 \item Per ciascuna delle invocazioni seguenti, predirre il risultato, quindi verificarlo nel framework:
\begin{itemize}
 \item \texttt{f3(1,2L)}
 \item \texttt{f4(1,2L)}
 \item \texttt{f3(1,1/2)}
 \item \texttt{f3(1,1/2f)}
 \item \texttt{f3(1,1/2)+3.4}
 \item \texttt{\textquotedbl{}\textquotedbl{}+f4(1,1/2f)}
\end{itemize}
\end{itemize}
\end{frame}

\begin{frame}
\frametitle{Correttezza dell'invocazione}
Dato il seguente gruppo di funzioni:
\begin{itemize}
 \item \texttt{int f(int x, int y)\{ return x;\}}
 \item \texttt{long f(int x, long y)\{ return y;\}}
 \item \texttt{boolean f(double x, double y)\{ return x==y;\}}
 \item \texttt{boolean f(float x, float y)\{ return x!=y;\}}
\end{itemize}
Si preveda (e si verifichi) il risultato delle valutazioni seguenti:
\begin{itemize}
 \item \texttt{f(0,1)}
 \item \texttt{f(0.0,1)}
 \item \texttt{f(0.0,1f)}
 \item \texttt{f(1/3,2l)}
 \item \texttt{f(0x01,0x10L)}
 \item \texttt{f(1/2,0.5)}
 \item \texttt{f(1/2,0.5f)}
 \item \texttt{f(2l,2l)}
 \item \texttt{f(010,07)}
\end{itemize}
\end{frame}

\begin{frame}
\frametitle{Ricorsione}
Le funzioni possono, al loro interno, chiamare altre funzioni o sé stesse. Si analizzi la funzione seguente: 
\begin{itemize}
 \item \texttt{int f(int n)\{ return n==0? 1 : n*f(n-1); \}}
\end{itemize}
\begin{itemize}
 \item Che tipo di comportamento realizza?
 \item Per quali input funziona come dovrebbe?
 \item Come si potrebbe migliorare, per far sì che funzioni correttamente per un insieme di input più ampio?
  \item La funzione effettua una ricorsione tail o non tail? Perché?
 \item Si costruisca l'albero delle invocazioni relativo alla chiamata \texttt{f(3)}.
\end{itemize}
\end{frame}

\begin{frame}[fragile]
\frametitle{Ricorsione}
Si analizzi la funzione \texttt{f(int, int)}:
\begin{verbatim}
int f(int x, int y) {
   return f(x, y, x, y);
}

int f(int x, int y, int z, int w) {
   return z==w ? z :
      z>w ? f(x, y, z, w+y) :
         f(x, y, z+x, w);
}
\end{verbatim}
\begin{itemize}
 \item Si costruisca l'albero delle invocazioni relativo alla chiamata \texttt{f(3,7)}.
 \item Che tipo di comportamento realizza? Si cerchi di intuirlo, in caso di difficoltà si usi il framework per ottenere dei suggerimenti.
  \item La funzione effettua una ricorsione tail o non tail? Perché?
\end{itemize}
\end{frame}

\begin{frame}[fragile]
\frametitle{Funzioni su array}
Si analizzi la funzione \texttt{sum(int[])}:
\begin{verbatim}
int sum(int[] array){
   return sum(array, 0);
}

int sum(int[] array, int pos){
   return (pos==array.length) ? 0 :
      array[pos] + sum(array, pos + 1);
}
\end{verbatim}
\begin{itemize}
 \item Quale funziona realizza? Una volta intuito, si utilizzi il framework per verificare.
 \item Si costruisca l'albero delle invocazioni relativo alla chiamata \texttt{sum(new int[]\{5,5,6,4\})}. 
\end{itemize}
\end{frame}

\begin{frame}[fragile]
\frametitle{Funzioni di test}
Si analizzi la funzione \texttt{test\_sum()}:
\begin{verbatim}
boolean test_sum(){
   return sum(new int[]{10,20,30})==60 &&
      sum(new int[]{10,20,30,40,50})==150 &&
      sum(new int[]{10})==10 &&
      sum(new int[]{})==0;
}
\end{verbatim}
\begin{itemize}
 \item Cosa realizza?
\end{itemize}
\end{frame}

\begin{frame}[fragile]
\frametitle{Modifica di funzioni}
\begin{itemize}
 \item Si modifichi la funzione \texttt{sum(int[])}, per fare in modo che, qualora l'array passato in ingresso abbia lunghezza superiore a 4, il risultato ritornato sia zero.
 \item Verificare tramite il framework il corretto funzionamento della nuova implementazione.
 \item Si modifichi \texttt{test\_sum()}, in maniera tale che venga provato il nuovo funzionamento di \texttt{sum(int[])}.
 \item Si modifichi \texttt{test\_sum()}, aggiungendo una verifica che provi il funzionamento di \texttt{sum(int[])} per array contenenti numeri negativi.
\end{itemize}
Si mostri al docente l'esercizio completato per ricevere commenti.
\end{frame}

\begin{frame}[fragile]
\frametitle{Modifica di funzioni}
Si consideri \texttt{sum(int[])}, nella versione non modificata.
\begin{itemize}
 \item Cosa succede se si sostituisce a ``\texttt{pos + 1}'' ``\texttt{pos + 2}''?
 \item Si costruisca l'albero delle invocazioni relativo alla chiamata \texttt{sum(new int[]\{10,2,30,4\})}
 \item Quale nuova funzione si realizza?
 \item Si costruisca l'albero delle invocazioni relativo alla chiamata \texttt{sum(new int[]\{10,2,30\})}, e lo si verifichi nel framework.
 \item Perché si ha un errore?
 \item Si sostituisca \texttt{==} con \texttt{>=}. Si cerchi di capire se questa modifica risolve il problema, e perché. Si verifichi nel framework.
\end{itemize}
\end{frame}

\begin{frame}[fragile]
\frametitle{Costruzione di funzioni a partire da funzioni esistenti}
Si consideri \texttt{sum(int[])}, nella versione non modificata.
\begin{itemize}
 \item A partire dalla soluzione precedente, si costruisca la funzione \texttt{sum\_odd(int[])}, che somma solo gli elementi in posizione dispari.
 \item Si scriva una funzione \texttt{test\_sum\_odd()}, che verifichi il funzionamento della precedente. In particolare, testare che funzioni con array di lunghezza zero, pari e dispari.
 \item Si costruisca la funzione \texttt{sum\_abs(int[])} che somma i valori assouti degli elementi dell'array.
 \item Si costruisca la funzione \texttt{test\_sum\_abs()} che verifica il corretto funzionamento di \texttt{sum\_abs(int[])} per array di varia lunghezza (fra cui zero), contenenti numeri positivi e negativi
\end{itemize}
Si mostri al docente l'esercizio completato per ricevere commenti.
\end{frame}

\begin{frame}[fragile]
\frametitle{Costruzione di funzioni a partire da funzioni esistenti: riconoscitori EBNF}
\begin{itemize}
 \item  \texttt{ric123(int[])} che riconosce la grammatica 1\{2\textbar{}3\}. Come esempio, si consideri il riconoscitore per 1\{2\} visto a lezione. Si scriva un test per verificarne il comportamento.
\end{itemize}
\begin{itemize}
 \item  Considerando il riconoscitore per la grammatica \{2\}1 sotto riportato, si realizzi \texttt{ric1234(int[])} che riconosce la grammatica \{1\textbar{}2\}34. Si scriva un test.
\end{itemize}
\scriptsize
\begin{verbatim}
boolean ric21(int[] a) {
   return ric2(a, 0);
}
boolean ric2(int[] a, int pos) {
   return pos == a.length ? false :
      a[pos] != 2 ? ric1(a, pos) : 
      ric2(a, pos + 1);
}
boolean ric1(int[] a, int pos) {
   return a[pos] == 1 && a.length == pos + 1;
}
\end{verbatim}
\normalsize
Si mostri al docente l'esercizio completato per ricevere commenti.
\end{frame}

\begin{frame}[fragile]
\frametitle{Esercizi aggiuntivi}
Da questo punto, vengono proposti esercizi aggiuntivi. Sono consigliati per coloro che hanno completato \textbf{tutti} i precedenti con successo e desiderano esercitarsi ulteriormente.
\end{frame}

\begin{frame}
\frametitle{Correttezza semantica, gruppi di funzioni}
Si noti che più funzioni possono avere lo stesso nome, a patto che la loro signature (senza considerare il tipo di ritorno) sia diversa. Si preveda quali delle seguenti coppie di funzioni possono essere definite contemporaneamente senza generare errori:
\begin{itemize}
 \item \texttt{int f(int x, int y)\{ return x+y;\}}
 \item \texttt{int f(long x, int y)\{ return x-y;\}}
\end{itemize}
\begin{itemize}
 \item \texttt{int f(int x, int y)\{ return x+y;\}}
 \item \texttt{int f(int x, int y)\{ return x-y;\}}
\end{itemize}
\begin{itemize}
 \item \texttt{long f(int x, int y)\{ return x+y;\}}
 \item \texttt{int f(int x, int y)\{ return x-y;\}}
\end{itemize}
\begin{itemize}
 \item \texttt{boolean f(double x, double y)\{ return x==y;\}}
 \item \texttt{boolean f(int x, int y)\{ return x==y;\}}
\end{itemize}
\end{frame}

\begin{frame}[fragile]
\frametitle{Ricorsione}
Si analizzi la funzione \texttt{f(int, int)}:
\begin{verbatim}
int f(int x) { return x == 0 ? 0 : f2(x-1); }
int f2(int x) { return x == 1 ? 1 : f(x-1); }
\end{verbatim}
\begin{itemize}
 \item Si costruisca l'albero delle invocazioni relativo alla chiamata \texttt{f(6)}, si verifichi il risultato con il framework.
 \item Si costruisca l'albero delle invocazioni relativo alla chiamata \texttt{f(5)}, si verifichi il risultato con il framework. 
\end{itemize}
\end{frame}

\begin{frame}[fragile]
\frametitle{Progettare nuove funzioni}
Si progettino le seguenti funzioni, e per ognuna si scriva un test appropriato che ne verifichi il funzionamento.
\begin{itemize}
 \item \texttt{int length(int[])}, che conta quanti elementi vi sono in un array.
 \item \texttt{int evens(int[])}, che conta quanti elementi pari vi sono in un array.
 \item \texttt{boolean arrayEquals(int[], int[])}, che ritorna true se i due array sono uguali (ossia hanno la stessa lunghezza, e presentano gli stessi elementi nello stesso ordine).
\end{itemize}
Si mostri al docente l'esercizio completato per ricevere commenti.
\end{frame}

\begin{frame}[fragile]
\frametitle{Progettare nuove funzioni: riconoscitori EBNF}
Si progettino le seguenti funzioni, e per ognuna si scriva un test appropriato che ne verifichi il funzionamento.
\begin{itemize}
 \item \texttt{ric111(int[])}, riconoscitore per la grammatica 1\{1\}1.
 \item \texttt{ric12(int[])}, riconoscitore per la grammatica \{1\textbar{}2\}.
 \item \texttt{ric122(int[])}, riconoscitore per la grammatica 1\{2\}3.
 \item \texttt{ric1234(int[])}, riconoscitore per la grammatica \{1\textbar{}2\}\{3\textbar{}4\}.
\end{itemize}
Si mostri al docente l'esercizio completato per ricevere commenti.
\end{frame}

% \begin{frame}[fragile]
% \frametitle{Avanzato: ottimizzazione tail}
% Le funzioni ricorsive tail sono ottimizzabili, e quindi potenzialmente più efficienti di quelle non-tail.
% \begin{itemize}
%  \item Si consideri la funzione \texttt{sum(int[])} vista in precedenza. La si modifichi facendo in modo che essa realizzi una ricorsione tail.
%  \item Si costruisca l'albero delle invocazioni relativo alla chiamata \texttt{sum(new int[]\{5,5,6,4\})} di questa nuova funzione. Lo si confronti con quello precedentemente costruito (per la versione non-tail): cosa cambia? 
%  \item Se le funzioni \texttt{sum\_odd(int[])} e \texttt{sum\_abs(int[])} precedentemente realizzate sono non-tail, se ne realizzi la versione tail.
% \end{itemize}
% \end{frame}

\begin{frame}
\frametitle{Al termine dell'esercitazione}
\begin{itemize}
 \item Ricordarsi di effettuare il logout
  \begin{itemize}
    \item Dal menu Start, selezionare ``Disconnetti''
  \end{itemize}
 \item Una volta disconnessi:
  \begin{itemize}
    \item Se siete del primo turno, lasciate le macchine accese, di modo che i vostri colleghi del turno successivo siano agevolati
    \item Se siete del secondo turno, spegnete le macchine
  \end{itemize}
\end{itemize}
\end{frame}


\end{document}

