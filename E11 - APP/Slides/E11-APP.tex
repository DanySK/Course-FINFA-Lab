%%%%%%%%%%%%%%%%%%%%%%%%%%%%%%%%%%%%%%%%%%%%%%%%%%%%%%%%%%%%%%%%%%%%%%%%%%%%%%%%
\documentclass{beamer}
%
\usepackage[italian]{babel}
\usepackage[utf8]{inputenc}
\usepackage[T1]{fontenc}
\newcommand{\vmfolder}{\texttt{C:\textbackslash{}VM\textbackslash{}Courses\textbackslash{}}}


%%%% Subtitle
%\subtitle[Short Subtitle]{Full Subtitle}
%%%% Authors
%%%% Institution/Partner
%%%% Date
% Structure slides
%
%%%%%%%%%%%%%%%%%%%%%%%%%%%%%%%%%%%%%%%%%%%%%%%%%%%%%%%%%%%%%%%%%%%%%%%%%%%%%%%%
\begin{document}
\title[Lab1 - FV]{Fondamenti di Informatica A \\ Laboratorio 11 \\ Applicazione completa}
\author[Danilo Pianini]{Danilo Pianini\\\texttt{danilo.pianini@unibo.it} \\ \vspace{3pt} Mirko Viroli\\\texttt{mirko.viroli@unibo.it} }
\institute[UNIBO]{\textsc{Alma Mater Studiorum}---Università di Bologna}
\date[\today]{\today}

\frame{\titlepage} 

\begin{frame}[fragile]
\frametitle{Operazioni preliminari}
\begin{itemize}
 \item \textbf{AVVISO:} per svolgere questo laboratorio è \textbf{necessario} utilizzare la macchina virtuale Linux del corso. La macchina è disponibile in \vmfolder{}.
  \begin{enumerate}
  \item Raggiungere la cartella \vmfolder{}
  \item Doppio click sul file \texttt{Courses.vbox} (icona blu)
  \item Si aprirà la finestra del gestore di macchine virtuali VirtualBox. Premere il tasto ``Avvia''
  \item Attendere l'avvio della macchina
 \end{enumerate}
 \item l'esercitazione è scaricabile da \url{http://campus.unibo.it/191166/}
 \item Solo \textit{dopo} aver completato tutti gli altri punti svolgete quelli contrassegnati come \textbf{OPZ.}: sono versioni leggermente più difficili degli esercizi precedenti.
\end{itemize}
\end{frame}

\begin{frame}[fragile]{Applicazione completa}
In questo laboratorio metterete mano dentro un'applicazione grafica C scritta da me.
\begin{itemize}
 \item Tale applicazione è in grado di mostrare a video delle funzioni, disegnandole all'interno di un'area di disegno.
 \item Inoltre, a ciascuna funzione può applicare un filtro che la modifica, e disegnare a video la funzione filtrata.
 \item Il codice è piuttosto commentato: aprite i sorgenti e cominciate a prendere dimestichezza con il sorgente
\end{itemize}
Componenti:
\begin{itemize}
 \item La libreria \texttt{colorfactory.h} che abbiamo visto a lezione è utilizzata anche qui
 \item La libreria \texttt{functions.h} descrive una struttura dati che associa un puntatore a funzione ad un nome
 \item La libreria \texttt{filters.h}, è molto simile, ma la struttura (parametri di ingresso e uscita) del puntatore a funzione è differente
 \item La libreria \texttt{points.h}, contiene una semplice struttura dati per la descrizione di un punto in 2D con coordinate cartesiane.
\end{itemize}
\end{frame}

\begin{frame}[fragile]{Compilazione}
Per prima cosa, bisogna compilare il programma. Vi ricordo che per compilare librerie che si appoggiano a GTK+3 è necessario utilizzare gcc come segue:
\scriptsize
\begin{verbatim}
gcc `pkg-config --libs --cflags gtk+-3.0` -c -Wall libreria.c
\end{verbatim}
\normalsize
Il particolare apice da utilizzare può essere ottenuto in ambiente Linux con la combinazione di tasti \texttt{AltGr+'}

Per compilare invece un programma che usa GTK+3 ed altre librerie (in questo caso \texttt{lib1} e \texttt{lib2}), il comando è:
\scriptsize
\begin{verbatim}
gcc `pkg-config --libs --cflags gtk+-3.0` lib1.o lib2.o -Wall programma.c
\end{verbatim}
\normalsize
\begin{itemize}
 \item Non è necessario compilare \texttt{points.h}, dato che non ha alcun file c di implementazione.
 \item Compilare \texttt{colorfactory.c} come libreria
 \item Compilare \texttt{functions.c} come libreria
 \item Compilare \texttt{filters.c} come libreria
 \item Compilare \texttt{FunctionDrawer.c} come programma che usa le tre librerie precedenti
\end{itemize}
\end{frame}

\begin{frame}[fragile]{Esecuzione}
Una volta compilato, si esegua l'applicazione tramite doppio click sull'eseguibile oppure lanciando il comando
\begin{verbatim}
./a.out
\end{verbatim}
Eventualmente sostituendo ad \texttt{a.out} il nome che avete dato all'eseguibile da produrre.

Provate a graficare diverse funzioni, applicare diversi filtri e a giocare con le sliders.
\end{frame}

\begin{frame}[fragile]{Cambiare colore al disegno della funzione filtrata}
La funzione filtrata è attualmente colorata in nero, il che la rende poco evidente. Si modifichi il programma affinché sia colorata in rosso.
\begin{enumerate}
 \item In \texttt{colorfactory.h}, si definisca un nuovo colore, ossia una variabile di tipo \texttt{const GdkRGBA*}, dandogli un nome appropriato (ad esempio, \texttt{color\_red})
 \item In \texttt{colorfactory.c}, si definisca una nuova funzione \\ \texttt{static GdkRGBA* gen\_color\_red(void)} \\ che inizializzi la variabile dichiarata nell'header. Si prendano come riferimento le tre funzioni fornite. Si ricorda, inoltre, che per creare il colore rosso è necessario mettere ad \texttt{1.0} i campi di colore che descrivono rosso e trasparenza (alpha) e a \texttt{0.0} i restanti campi.
 \item In \texttt{colorfactory.c}, si modifichi la funzione \texttt{void init\_color\_factory(void)} in maniera tale che inizializzi anche il nuovo colore. Si prenda spunto dal modo in cui vengono inizializzati gli altri colori.
\end{enumerate}
\end{frame}

\begin{frame}[fragile]{Cambiare colore al disegno della funzione filtrata}
\begin{enumerate}
\setcounter{enumi}{3}
 \item Si ricompili colorfactory.c
 \item In \texttt{FunctionDrawer.c}, nella funzione \\ \texttt{static void draw\_function(cairo\_t *)}\\ si modifichi l'argomento passato alla funzione \texttt{gdk\_cairo\_set\_source\_rgba} quando viene chiamata per definire il colore della funzione filtrata, in maniera tale che il colore non sia più \texttt{color\_black} ma il nuovo colore che avete definito.
 \item Si ricompili \texttt{FunctionDrawer.c} e si verifichi che la funzione filtrata appaia ora di colore rosso.
\end{enumerate}
\end{frame}

\begin{frame}[fragile]{Implementazione di nuove funzioni}
Si implementino nel programma le seguenti nuove funzioni:
\begin{itemize}
 \item \texttt{``cos(x)''} --- $\cos(x)$
 \item \texttt{``log(x)''} --- $\log(x)$
 \item \textbf{OPZ.} \texttt{``wing(x)''} --- $\left\{
        \begin{array}{ll}
            \cos(x) + 5 \log(-x + 1) & \quad x \leq 0 \\
            \cos(x) + 5 \log(x + 1) & \quad x > 0
        \end{array}
    \right.$
\end{itemize}
Per implementare una qualunque nuova funzione \texttt{f}, si seguano i seguenti passi:
\begin{enumerate}
 \item In \texttt{functions.c}, prendendo spunto dalle implementazioni esistenti, si implementi \\ \texttt{static double f(double)}
 \item In \texttt{functions.c}, si modifichi la funzione \\ \texttt{void init\_functions(void)} \\ in maniera che la variabile \texttt{functions\_number} contenga il corretto numero di funzioni definite, e che la nuova funzione \texttt{f} abbia una propria entry nell'array \texttt{functions}. Si osservi come vengono gestite le funzioni esistenti.
\end{enumerate}
\end{frame}

\begin{frame}[fragile]{Implementazione di nuove funzioni}
\begin{enumerate}
\setcounter{enumi}{2}
 \item Si ricompili \texttt{functions.c}
 \item Si ricompili \texttt{FunctionDrawer.c} e si verifichi che la funzione aggiunta appare ora nell'elenco di quelle disponibili. Si verifichi che venga graficato quanto atteso.
\end{enumerate}
\end{frame}

\begin{frame}[fragile]{Implementazione di nuovi filtri}
Si implementino nel programma i seguenti nuovi filtri:
\begin{itemize}
 \item \texttt{absolute} --- calcola la funzione modulo della funzione data ($\sqrt{(f(x))^2}$), ossia trasforma in positivi i valori negativi. Si prenda spunto dal filtro \texttt{onlypos} già implementato, che azzera i valori negativi di una funzione.
 \item \textbf{OPZ.} \texttt{derivate} --- calcola la funzione derivata della funzione data $f'(x)$. Si ricordi la definizione di derivata come limite del rapporto incrementale: il valore della derivata è approssimabile come $f'(x) \approx \frac{f(x + \varepsilon) - f(x)}{\varepsilon}$. Si prenda spunto dal filtro \texttt{integrate} già implementato, che calcola (a meno di una costante) il valore della funzione integrale della funzione data ($\int{f(x)dx}$), approssimando l'integrale come somma dell'area di rettangoli.
\end{itemize}
\end{frame}

\begin{frame}[fragile]{Implementazione di nuovi filtri}
Per implementare un qualunque nuovo filtro \texttt{f}, si seguano i seguenti passi:
\begin{enumerate}
 \item In \texttt{filters.c}, prendendo spunto dalle implementazioni esistenti, si implementi \\ \texttt{void f(GArray *, GArray *)}
 \item In \texttt{filters.c}, si modifichi la funzione \\ \texttt{void init\_filters(void)} \\ in maniera che la variabile \texttt{filters\_number} contenga il corretto numero di filtri definiti, e che il nuovo filtro \texttt{f} abbia una propria entry nell'array \texttt{filters}. Si osservi come vengono gestiti i filtri esistenti.
 \item Si ricompili \texttt{filters.c}
 \item Si ricompili \texttt{FunctionDrawer.c} e si verifichi che il filtro aggiunto appare ora nell'elenco di quelli disponibili. Si verifichi che venga graficato quanto atteso.
\end{enumerate}
\end{frame}

\begin{frame}[fragile]{Creazione di un pulsante}
Si crei un pulsante nell'interfaccia la cui funzione sia quella di salvare su disco i punti calcolati dalla funzione filtrata. Per farlo, si seguano i seguenti passi:
\begin{enumerate}
 \item In \texttt{FunctionDrawer.c}, all'interno della funzione \\ \texttt{static void activate(GtkApplication *, gpointer)} \\ si definisca un nuovo \texttt{GtkWidget *}. Lo si chiami, ad esempio, \texttt{save\_button}. Tale definizione può avvenire assieme a quelle degli altri \texttt{GtkWidget *} già presenti.
 \item In \texttt{FunctionDrawer.c}, all'interno della funzione \\ \texttt{static void activate(GtkApplication *, gpointer)}, si istanzi il nuovo pulsante utilizzando la funzione\\ \texttt{GtkWidget *gtk\_button\_new\_with\_label(char *)}\\ Una corretta invocazione può essere ad esempio: \\ \texttt{save\_button = gtk\_button\_new\_with\_label(\textquotedbl{}Save\textquotedbl{});}
\end{enumerate}
\end{frame}

\begin{frame}[fragile]{Aggiunta del pulsante, collegamento della callback}
\begin{enumerate}
\setcounter{enumi}{2}
 \item Si aggiunga il pulsante al layout verticale di destra, in fondo. Per farlo, si invochi la funzione seguente: \\ \texttt{gtk\_box\_pack\_start(GTK\_BOX(rlayout), save\_button, TRUE, FALSE, 0);} \\ Se si vuole posizionare il pulsante in basso, tale chiamata deve essere posta dopo le altre invocazioni della stessa funzione.
 \item Si compili \texttt{FunctionDrawer.c} e si verifichi che il pulsante (ancora privo di funzionalità collegate ai propri signals) sia presente nell'interfaccia dove desiderato.
 \item Si colleghi la funzione callback \\ \texttt{void save(GtkApplication *, gpointer)} \\ già implementata al signal ``clicked'' del pulsante. Per farlo, si utilizzi la seguente linea di codice: \\ \texttt{g\_signal\_connect(save\_button, \textquotedbl{}clicked\textquotedbl{}, G\_CALLBACK(save), NULL);}
 \item Si compili \texttt{FunctionDrawer.c} e si verifichi che il pulsante quando premuto ora apre una finestra di salvataggio.
\end{enumerate}
\end{frame}

\begin{frame}[fragile]{Completamento della callback, salvataggio dati}
\begin{enumerate}
\setcounter{enumi}{6}
 \item La callback \\ \texttt{void save(GtkApplication *, gpointer)} \\ sfortunatamente non è completa: manca la parte di salvataggio dei dati.
 \item Si utilizzi \texttt{fprinf} per salvare tutti i dati contenuti nel \texttt{GArray filtered}. Per farlo, si scorra filtered e per ciascun punto in esso contenuto si stampi su file la coordinata x, uno spazio, la coordinata y, e un ``a capo''.
 \item Come riferimento, si consideri che la funzione \\ \texttt{static void load(void)} \\ già implementata deve essere in grado di caricare correttamente i punti salvati salla \texttt{save}.
 \item Si compili \texttt{FunctionDrawer.c} e si verifichi che il pulsante quando premuto ora apre una finestra di salvataggio, e che a fronte della selezione di un file esso viene riempito con i dati come atteso.
 \item Si verifichi che il programma è in grado di caricare i punti salvati in precedenza.
\end{enumerate}
\end{frame}

\begin{frame}[fragile]{Fine dei giochi}
Al termine dei laboratori di questo corso dovreste essere in grado di:
\begin{itemize}
 \item Scrivere programmi giocattolo in Java
 \item Scrivere semplici programmi in C
 \item Scrivere nuove librerie C
 \item Utilizzare librerie C scritte da altri
 \item Realizzare programmi C con grafica e accesso al file system (conseguenza della precedente)
 \item Last but not least, \textbf{passare l'esame}!
 \item Chi ha trovato il corso interessante e volesse imparare a scrivere software ``sul serio'', ha le basi necessarie per affrontare il corso di Programmazione a Oggetti da 12 crediti a Ingegneria e Scienze informatiche. Non essendo più fra gli opzionali, richiede una modifica del piano di studi.
\end{itemize}
\end{frame}



\end{document}

