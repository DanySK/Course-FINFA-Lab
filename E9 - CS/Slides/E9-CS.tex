%%%%%%%%%%%%%%%%%%%%%%%%%%%%%%%%%%%%%%%%%%%%%%%%%%%%%%%%%%%%%%%%%%%%%%%%%%%%%%%%
\documentclass{beamer}
%
\usepackage[italian]{babel}
\usepackage[utf8]{inputenc}
\usepackage[T1]{fontenc}

%%%% Subtitle
%\subtitle[Short Subtitle]{Full Subtitle}
%%%% Authors
%%%% Institution/Partner
%%%% Date
% Structure slides
%
%%%%%%%%%%%%%%%%%%%%%%%%%%%%%%%%%%%%%%%%%%%%%%%%%%%%%%%%%%%%%%%%%%%%%%%%%%%%%%%%
\begin{document}
\title[Lab1 - FV]{Fondamenti di Informatica A \\ Laboratorio 9 \\ Strutture dati in C}
\author[Danilo Pianini]{Danilo Pianini\\\texttt{danilo.pianini@unibo.it} \\ \vspace{3pt} Mirko Viroli\\\texttt{mirko.viroli@unibo.it} }
\institute[UNIBO]{\textsc{Alma Mater Studiorum}---Università di Bologna}
\date[\today]{\today}

\frame{\titlepage} 

\begin{frame}[fragile]
\frametitle{Operazioni preliminari}
\begin{itemize}
 \item Accendere il computer
 \item Effettuare il log-in con le proprie credenziali istituzionali
  \begin{itemize}
    \item Normalmente sono simili a: \texttt{nome.cognome@studio.unibo.it}
  \end{itemize}
 \item Aprire Google Chrome, o altro browser
 \item Collegarsi al sito \url{http://bit.ly/finfa2013-e9}
 \item Scaricare il pacchetto \url{http://bit.ly/finfa2013-e9-code}
 \item Scompattare il pacchetto in una cartella a piacimento: questa sarà la vostra directory di lavoro di oggi.
\end{itemize}
\end{frame}

\begin{frame}[fragile]
\frametitle{\texttt{struct} e \texttt{typedef}}
Si consideri il contenuto della cartella \texttt{es0}
\begin{itemize}
 \item Si osservi con attenzione \texttt{point\_base.h}. Cosa descrive?
 \item Si osservi con attenzione \texttt{use\_point\_base.c}. Solo dopo averne compreso appieno il funzionamento, si compili e si esegua.
 \item Si osservi ora \texttt{point.h}. In cosa differisce da \texttt{point\_base.h}?
 \item Si osservi con attenzione \texttt{use\_point.c}. Qual è il risvolto pratico dell'uso della keyword \texttt{typedef}? Solo dopo averne compreso appieno il funzionamento, si compili e si esegua.
\end{itemize}
\end{frame}

\begin{frame}[fragile]
\frametitle{Una libreria C}
Si consideri il contenuto della cartella \texttt{es1}.
\begin{itemize}
 \item \texttt{complex.c} implementa una libreria di operazioni sui numeri complessi
 \item complex.h è un file header che contiene la definizione della struttura dati per rappresentare numeri complessi e il prototipo di tutte le funzioni sui complessi implementate dalla libreria.
 \item usecomplex.c contiene un main che effettua alcune prove di funzionamento sulla libreria.
 \item Si comprenda il funzionamento della libreria, si compili e si esegua.
\end{itemize}
\end{frame}

\begin{frame}[fragile]
\frametitle{Esportazione di una libreria C}
Data la libreria di funzioni sulle liste definita nel file \texttt{list.c} (all'interno della cartella \texttt{es2}), in questa seconda parte dell'esercitazione si dovrà rendere disponibile la libreria ad altri file C, definendo opportunamente un file header per la libreria.
\begin{itemize}
 \item Per prima cosa, creare un nuovo file \texttt{list.h}, dove secondo lo stesso schema del file header per i complessi \texttt{complex.h}, definirete:
 \begin{enumerate}
  \item la struttura dati rappresentante una lista (spostandola nell'header dal file \texttt{list.c})
  \item la definizione dei tipo \texttt{List} (istruzione \texttt{typedef} anch'essa da togliere da \texttt{list.c} e spostare in \texttt{list.h})
  \item il prototipo di tutte le funzioni implementate da \texttt{list.c}
 \end{enumerate}
 \item \textbf{SUGGERIMENTO}: trarre spunto dalla struttura utilizzata per il file di intestazione della libreria sui numeri complessi.
 \item compilare la libreria per le liste, generando in uscita il file oggetto \texttt{list.o}
\end{itemize}
\end{frame}

\begin{frame}[fragile]
\frametitle{Uso della libreria esportata}
Definire un file \texttt{uselist.c} contenente un main nel quale realizzare i seguenti test
\begin{itemize}
 \item Creazione di una lista (10,20,30,40):
\begin{enumerate}
 \item creare un array di interi contenente i 4 elementi specificati
 \item creare la lista utilizzando la funzione \\ \texttt{List *list\_from\_array(int [],int)} al quale passare in ingresso l'array di interi appena creato, e che restituirà in uscita il puntatore alla lista appena creata
 \item Assegnare il puntatore a una variabile di tipo \texttt{List *};
\end{enumerate}
 \item Creare ora una lista (50,60), questa volta utilizzando la funzione \texttt{List *list\_cons(int, List *)}, funzione che rappresenta il costruttore di una lista e che accetta in ingresso un intero e un puntatore ad una lista.
 \item \textbf{SUGGERIMENTO}: utilizzare il costruttore ricorsivamente! Ad esempio per creare la lista (1,2,3) utilizzare il costruttore nel seguente modo: \texttt{cons(1,cons(2,cons(3,nil)));} ricordare che la lista vuota può essere creata tramite la funzione \texttt{List *list\_nil(void)}.
\end{itemize}
\end{frame}

\begin{frame}[fragile]
\frametitle{Uso della libreria esportata}
\begin{itemize}
 \item Stampare su console la rappresentazione delle due liste appena create utilizzando la funzione \texttt{char *list\_to\_string(List *)} che accetta in ingresso il puntatore a una lista e restituisce l'array di caratteri contenente la stringa rappresentante la lista. \item \textbf{SUGGERIMENTO}: per stampare la lista su console utilizzare la funzione \texttt{printf} e il parametro \texttt{\%s}, come visto più volte a lezione.
 \item Stampare su console la dimensione delle due liste avvalendosi della funzione \texttt{int list\_length(List *)}.
 \item Ora "appendere" alla lista la la seconda lista lb: utilizzare la funzione \texttt{void list\_append\_to(List *,List *)} nel seguente modo: \texttt{list\_append\_to(la,lb)}. In coda alla lista \texttt{la} si troverà ora la lista \texttt{lb}: accertarsene stampando su console la stringa rappresentante la lista e la sua dimensione.
 \item Compilare ed eseguire.
\end{itemize}
\end{frame}

\begin{frame}[fragile]
\frametitle{Uso degli argomenti del programma}
\begin{itemize}
 \item Così come in Java, anche in C è possibile passare degli argomenti ad un programma. Per utilizzarli, il main deve avere la signature: \\ \texttt{int main(int, char **)}\\ Il primo parametro che viene inserito è il numero di elementi presenti nel secondo argomento, che rappresenta il comando che è stato eseguito nel suo intero.
 \item Si osservi il file \texttt{echo.c} nella cartella \texttt{es3}. Cosa realizza? Si compili e si testi il programma, invocando ad esempio \\\texttt{./echo mi piace informatica}
 \item Cosa accade se la variabile i nel main di echo.c viene inizializzata a 0 e non a 1? Si eseguano delle prove.
\end{itemize}
\end{frame}

\begin{frame}[fragile]
\frametitle{Conversione da stringhe a numeri}
Gli argomenti passati sono sempre in formato stringa (char *). Così come Java, C mette a disposizione delle utility di libreria in \texttt{stdlib.h} per convertire tali stringhe in numeri (\texttt{int}, \texttt{long}, \texttt{double}) in modo da poterli usare internamente. Tali funzioni sono:
\begin{enumerate}
 \item \texttt{int atoi (const char * str)} --- Converte in \texttt{int}
 \item \texttt{long int atol (const char * str)} --- Converte in \texttt{long}
 \item \texttt{double atof (const char* str)} --- Converte in \texttt{double}
\end{enumerate}
\begin{itemize}
 \item Si osservi il contenuto di \texttt{input.c}. Cosa realizza?
 \item Si compili \texttt{input.c} e si provi ad utilizzare il programma con diversi argomenti. Cosa accade se si passa un letterale?
 \item Prendendo spunto da \texttt{input.c}, si realizzi in \texttt{sum.c} un programma che, dati in ingresso un numero arbitrario di argomenti di tipo intero, ne stampi a video la somma.
\end{itemize}
\end{frame}

\begin{frame}[fragile]
\frametitle{Puntatori a funzione}
In C è possibile passare una funzione come argomento di una funzione. Si consideri il contenuto della cartella \texttt{es4}.
\begin{itemize}
 \item Si osservi il contenuto di \texttt{functptr.c}. Cosa realizza?
 \item Si compili e si esegua \texttt{functptr.c}.
\end{itemize}
\end{frame}

\begin{frame}[fragile]
\frametitle{Puntatori a funzione}
Si modifichi la libreria delle liste creata prima come segue:
\begin{itemize}
 \item In \texttt{list.h}, definisca una nuova funzione\\ \texttt{void apply\_to\_all(List *, void (int *))}\\ che torna \texttt{void} e prende in ingresso
 \begin{enumerate}
  \scriptsize
  \item un puntatore ad una \texttt{List}
  \item una funzione che prende in ingresso un puntatore ad \texttt{int} e torna \texttt{void}
 \end{enumerate}
 E applica la funzione f a tutti gli elementi della lista
 \item Si implementi in \texttt{list.c} tale funzione
 \scriptsize
 \item \textbf{SUGGERIMENTO}: Si scorra la lista fintanto che non si trova la lista vuota, e si applichi la funzione f al puntatore alla testa dell'elemento corrente.
 \normalsize
 \item Si utilizzi la versione modificata di \texttt{uselist.c} fornita nella cartella \texttt{es4}
 \item Si definisca all'interno di \texttt{uselist.c} una funzione \\ \texttt{void filter(int *)} che, dato in ingresso un puntatore ad un intero, modifica l'elemento puntato nel seguente modo:
 \begin{itemize}
  \item Se l'elemento è pari, lo incrementa di 1
  \item Se l'elemento è dispari, lo azzera
 \end{itemize}
 \item Si osservi attentamente come viene utilizzata la \texttt{apply\_to\_all} nel \texttt{main}. Si compili e si esegua.
\end{itemize}
\end{frame}

\begin{frame}[fragile]
\frametitle{Esercizi aggiuntivi}
Come al solito, se terminate in anticipo tutti gli esercizi precedenti, vi consigliamo di allenarvi con quelli nella cartella \texttt{aggiuntivi}.
\end{frame}




\end{document}

