%%%%%%%%%%%%%%%%%%%%%%%%%%%%%%%%%%%%%%%%%%%%%%%%%%%%%%%%%%%%%%%%%%%%%%%%%%%%%%%%
\documentclass{beamer}
%
\usepackage[italian]{babel}
\usepackage[utf8]{inputenc}
\usepackage[T1]{fontenc}

%%%% Subtitle
%\subtitle[Short Subtitle]{Full Subtitle}
%%%% Authors
%%%% Institution/Partner
%%%% Date
% Structure slides
%
%%%%%%%%%%%%%%%%%%%%%%%%%%%%%%%%%%%%%%%%%%%%%%%%%%%%%%%%%%%%%%%%%%%%%%%%%%%%%%%%
\begin{document}
\title[Lab1 - FV]{Fondamenti di Informatica A \\ Laboratorio 1 \\ Valutazione e typing di espressioni}
\author[Danilo Pianini]{Danilo Pianini\\\texttt{danilo.pianini@unibo.it} \\ \vspace{3pt} Mirko Viroli\\\texttt{mirko.viroli@unibo.it} }
\institute[UNIBO]{\textsc{Alma Mater Studiorum}---Università di Bologna}
\date[\today]{\today}

\frame{\titlepage} 

\begin{frame}
\frametitle{Operazioni preliminari}
\begin{itemize}
 \item Accendere il computer
 \item Effettuare il log-in con le proprie credenziali istituzionali
  \begin{itemize}
    \item Normalmente sono simili a: \texttt{nome.cognome@studio.unibo.it}
  \end{itemize}
 \item Aprire Google Chrome, o altro browser
 \item Collegarsi al sito \url{http://bit.ly/finfa2013-e1}
  \begin{itemize}
    \item in ``e1'', il carattere ``1'' è un uno, non una L minuscola!
  \end{itemize}
 \item Scaricare l'esercitazione odierna
\end{itemize}
\end{frame}

\begin{frame}
\frametitle{Verifica del funzionamento del Framework F-V}
\begin{itemize}
 \item Sul desktop, troverete la cartella ``framework''
 \item Fate doppio click sulla cartella
 \item Si aprirà una nuova finestra, all'interno della quale troverete tre software
 \item Fare doppio click sull'icona del FrameworkFV
 \item Inserire nel campo ``Espressioni da valutare'' il testo \texttt{$1+2$}
 \item Premere il tasto ``Valutazione''
 \item Verificare che il risultato sia ``3''
 \item In caso di risultato diverso dalle attese, chiamare subito il docente o il tutor
\end{itemize}
\end{frame}

\begin{frame}
\frametitle{Come esercitarsi con il Framework}
\begin{itemize}
 \item Nelle slides seguenti vi verranno proposti una serie di esempi di espressioni da valutare
 \item \textbf{Prima} di eseguirle nel programma, cercare \textbf{sempre} di anticipare quale sarà il risultato della valutazione
 \item Se quanto avete pensato corrisponde a quanto risponde il software, bene!
 \item Se non corrisponde, bene lo stesso, a patto che ne capiate la ragione
 \item Se ottenete un risultato inaspettato e non sapete giustificarlo, consultate il docente!
 \item Al termine dell'esercitazione, \textbf{effettuate il log out} dalla macchina che state utilizzando.
\end{itemize}
\end{frame}

\begin{frame}
\frametitle{Come comportarsi di fronte agli errori}
\begin{itemize}
 \item Errare humanum est
  \begin{itemize}
    \item Ed è anche molto meglio sbagliare in laboratorio che all'esame!
  \end{itemize}
 \item È importante, di fronte agli errori, non scoraggiarsi, leggere \textbf{attentamente} il messaggio di errore, \textbf{capirlo} e studiare una contromisura.
 \item Java segnala gli errori in maniera piuttosto precisa...
 \item ...il framework, che ne usa una sottoparte molto ristretta, molto meno!
 \item In caso di errore, il framework vi mostrerà un messaggio di errore, seguito dal codice Java reale che è stato invocato.
\end{itemize}
\end{frame}

\begin{frame}
\frametitle{Operazioni su booleani}
\begin{itemize}
 \item \texttt{true}
 \item \texttt{false}
 \item \texttt{!false}
 \item \texttt{true \& false}
 \item \texttt{true \textbar{} false}
 \item \texttt{true \textbar{} true \& false \textbar{} false} --- Quale operatore è prioritario? In che ordine si valuta?
 \item \texttt{true \& true \textbar{} false \& false}
 \item \texttt{true \& (true \textbar{} false) \& false}
\end{itemize}
\end{frame}

\begin{frame}
\frametitle{Operazioni su interi -- elementi di base}
\begin{itemize}
 \item \texttt{1+2}
 \item \texttt{5*8}
 \item \texttt{10/2}
 \item \texttt{10/3} --- Quanto fa? $3$, $3.\bar{3}$ o $4$?
 \item \texttt{10\%3} --- Classi di resto, come in Geometria
 \item \texttt{6/2*(1+2)} --- Quanto fa? $6$ o $9$?
 \item \texttt{4+4*4+4*4-4/4+4*4}
\end{itemize}
\end{frame}

\begin{frame}
\frametitle{Operazioni su interi -- complemento a 2, overflow}
In Java, gli interi negativi sono rappresentati in complemento a 2. Altri linguaggi, come C, consentono di dichiarare \texttt{unsigned} alcuni tipi di dato, ossia mai negativi. Java non offre questa possibilità. Calcolate \textbf{a mano} il complemento a due \textbf{prima} di eseguire!
\begin{itemize}
 \item \texttt{8-7} --- Trasformate 7 in complemento a 2, quindi sommate!
 \item \texttt{2147483647+1} --- Cos'è successo??? Suggerimento: in Java gli \texttt{int} si rappresentano con \textbf{32bit}!
 \item \texttt{2147483647*2}
 \item \texttt{-2147483648-1}
 \item \texttt{-(-2147483648)}
 \item \texttt{-(-2147483648L)} --- La \texttt{L} finale indica a Java di interpretare quel numero come tipo \texttt{long}. I \texttt{long}, a differenza degli \texttt{int}, hanno \textbf{64bit}.
\end{itemize}
\end{frame}

\begin{frame}
\frametitle{Operazioni su numeri in virgola mobile}
Java (così come C) consente di operare con numeri decimali. Ne esistono di due tipi \texttt{float}, rappresentati con \textbf{32bit}, e \texttt{double}, rappresentati con \textbf{64bit}.
\begin{itemize}
 \item \texttt{1.0} --- Aggiungere una parte decimale ad un numero lo fa interpretare a Java come \texttt{double}.
 \item \texttt{1+2.1} --- Un'operazione fra un \texttt{int} e un \texttt{double} ha come risultato un double
 \item \texttt{10.0/3} --- Confrontare il risultato con \texttt{10/3}
 \item \texttt{10d/3} --- Posporre \texttt{d} ad un numero forza Java ad interpretarlo come \texttt{double}.
 \item \texttt{10f/3} --- Posporre \texttt{f} ad un numero forza Java ad interpretarlo come \texttt{float}. Notare come i \texttt{float} siano meno precisi dei \texttt{double}, confrontando questo risultato con il precedente
 \item \texttt{15.45E24} --- Supporto alla notazione scientifica
\end{itemize}
\end{frame}

\begin{frame}
\frametitle{Precisione dei numeri in virgola mobile}
Internamente, sia i double che i float, riservano alcuni bit alla rappresentazione dell'esponente, ed altri alla rappresentazione della parte significativa (mantissa): esattamente come i numeri scritti in notazione scientifica.
\begin{itemize}
 \item \texttt{1e10+1e-5}
 \item \texttt{1e10+1e-6} --- Troppe cifre significative per rappresentare correttamente il numero!
 \item \texttt{1e10+1e-7} --- I numeri sono di ordine tanto diverso che a fronte della somma il secondo viene ``mangiato''.
 \item \texttt{1+1e-15}
 \item \texttt{1+1e-15*4} --- La moltiplicazione viene fatta per prima, dunque non si ha perdita di precisione.
 \item \texttt{1+1e-15*3+1e-15}
 \item \texttt{4+1e-15+1e-15+1e-15+1e-15} 
\end{itemize}
\end{frame}

\begin{frame}
\frametitle{Operazioni su stringhe}
Java consente di manipolare anche le stringhe con l'operatore ``+''. Questa caratteristica è assente o differente in molti altri linguaggi (in C ad esempio).
\begin{itemize}
 \item \texttt{\textquotedbl{}a\textquotedbl{}+\textquotedbl{}b\textquotedbl{}}
 \item \texttt{\textquotedbl{}1\textquotedbl{}+\textquotedbl{}2\textquotedbl{}} --- Le stringhe non vengono convertite in numeri automaticamente!
 \item \texttt{1+\textquotedbl{}2\textquotedbl{}} --- La somma di un numero e una stringa, ha come risultato una stringa
 \item \texttt{1+2+\textquotedbl{}3\textquotedbl{}+4+5}
 \item \texttt{1+2+\textquotedbl{}3\textquotedbl{}+\textquotedbl{}4\textquotedbl{}+\textquotedbl{}5\textquotedbl{}}
\end{itemize}
\end{frame}

\begin{frame}
\frametitle{Arrays}
\begin{itemize}
 \item \texttt{new int[]\{\}} --- Tipo per \texttt{riferimento}!
 \item \texttt{new int[]\{\}.length}
 \item \texttt{new int[]\{1, 2, 3\}}
 \item \texttt{new int[10].length}
 \item \texttt{new boolean[]\{true, true, false\}[1]}
 \item \texttt{new String[]\{}``\texttt{a}''\texttt{,}``\texttt{b}''\texttt{\}[0]}
\end{itemize}
\end{frame}

% \begin{frame}
% \frametitle{Formato Binario, Ottale, Esadecimale}
% Java offre delle utilità per formattare i numeri in basi diverse. Queste sono piuttosto comode per capire cosa accada, ma \textbf{non} usatele quando vi esercitate, dato che all'esame \textbf{non} potrete farne uso!
% \begin{itemize}
%  \item \texttt{Integer.toBinaryString(2147483647)} 
%  \item \texttt{Integer.toBinaryString(2147483647+1)} 
%  \item \texttt{Integer.toBinaryString(-(-2147483648))} 
%  \item \texttt{Integer.toHexString(2147483647)} 
%  \item \texttt{Integer.toHexString(2147483647+1)} 
%  \item \texttt{Integer.toHexString(-(-2147483648))}
% \end{itemize}
% Usatele per gli esercizi della prossima slide se qualcosa non vi torna, ma solo \textbf{dopo} aver fatto i conti a mano.
% 
% \textbf{Attenzione:} non vengono inseriti gli zeri a sinistra per completare il numero di bit! Attenti ai segni!
% \end{frame}

\begin{frame}
\frametitle{Operatori bitwise: shift}
Java (come C) consente di operare anche a livello di bit. %Usare le utilità prima presentate per verificare i risultati.
\begin{itemize}
 \item \texttt{$1<<0$} 
 \item \texttt{$1<<3$} 
 \item \texttt{$1<<32$} 
 \item \texttt{$(1<<2)<<3$} 
 \item \texttt{$0xf << 4$} 
 \item \texttt{$0xabcd << 8$} 
 \item \texttt{$100 >>> 1$} --- Non considera il segno!
 \item \texttt{$-100 >> 1$} 
 \item \texttt{$-100 >>> 1$} --- Visto?
\end{itemize}
\end{frame}

\begin{frame}
\frametitle{Operatori bitwise: \& e \textbar{}}
Java (come C) consente di operare anche a livello di bit.% Usare le utilità prima presentate per verificare i risultati.
\begin{itemize}
 \item \texttt{1 \textbar{} 2} 
 \item \texttt{1 \textbar{} 4} 
 \item \texttt{12345 \& 54321} 
 \item \texttt{-12345 \& 54321} 
 \item \texttt{12345 \& -54321} 
 \item \texttt{-12345 \& -54321} 
 \item \texttt{1$<<$7 \textbar{} 1$<<$10 } 
 \item \texttt{0xffffffff \& 0x0000ffff} --- L'operatore \& è spesso usato per ``mascherare'' una parte dati considerata non significativa
 \item \texttt{0xffffffff \& 0xffff0000} 
 \item \texttt{0xf$<<$12 \textbar{} 0xf$<<$4} 
 \item \texttt{1$<<$12 \textbar{} 2$<<$12 \textbar{} 4$<<$12 \textbar{} 8$<<$12} 
\end{itemize}
\end{frame}

\begin{frame}
\frametitle{Errori sintattici}
Provare queste espressioni ed osservare con cura il tipo di errore!
\begin{itemize}
 \item \texttt{1+>3} 
 \item \texttt{1+(2} 
 \item \texttt{1e} 
 \item \texttt{1e/5} 
 \item \texttt{1)} --- Quale errore è segnalato?
 \item \texttt{1^^4} 
 \item \texttt{\textquotedbl{}a}
\end{itemize}
Provate a commettere altri errori di sintassi, e verificate il comportamento del framework ed il tipo di errore ritornato!
\end{frame}

\begin{frame}
\frametitle{Errori semantici}
Provare queste espressioni ed osservare con cura il tipo di errore!
\begin{itemize}
 \item \texttt{1/0} 
 \item \texttt{1.0/0} --- I numeri in virgola mobile supportano $\infty$!
 \item \texttt{-1.0/0} 
 \item \texttt{1\%0}
 \item \texttt{1.0\%0} --- Cosa succede?
 \item \texttt{2147483648} --- Come si più risolvere?
 \item \texttt{1+false} 
 \item \texttt{a} 
 \item \texttt{new int[]\{true\}} 
 \item \texttt{new int[]\{1\}[a]}
 \item \texttt{new int[]\{1, 2\}[10]}
 \item \texttt{new int[]\{1, 2\}[-1]}
 \item \texttt{new int[]\{1, 2\}[2]} --- Perché dà errore?
\end{itemize}
\end{frame}

\begin{frame}
\frametitle{Esercizio riassuntivo}
Considerando le priorità degli operatori viste a lezione, si risolva su carta e si provi la correttezza tramite il framework FV di:
\begin{itemize}
 \item \texttt{5+7\&8 \textbar{} 9 \textbar{} 10$<<$3-$\sim$7} 
  \begin{itemize}
  \item In Windows, il carattere ``$\sim$'' si ottiene tenendo premuto Alt e premendo in sequenza 1, 2, 6 nel tastierino numerico.
  \end{itemize}
\end{itemize}
\end{frame}

\begin{frame}
\frametitle{Al termine dell'esercitazione}
\begin{itemize}
 \item Ricordarsi di effettuare il logout
  \begin{itemize}
    \item Dal menu Start, selezionare ``Disconnetti''
  \end{itemize}
 \item Una volta disconnessi:
  \begin{itemize}
    \item Se siete del primo turno, lasciate le macchine accese, di modo che i vostri colleghi del turno successivo siano agevolati
    \item Se siete del secondo turno, spegnete le macchine
  \end{itemize}
\end{itemize}
\end{frame}


\end{document}

